
% SPDX-License-Identifier: GPL-2.0+

% Copyright (C) 1996-2024  Jacek Kobus 

\documentclass[10pt,a4paper]{article}
\hoffset -1.0cm
\voffset -0.5in
\textwidth 15.0cm
\textheight 23.0cm
\usepackage[hidelinks]{hyperref} 
\newcommand{\beq}{\begin{equation}}
\newcommand{\eeq}{\end{equation}}
\newcommand{\cl} {\centerline}
\newcommand{\bde}{\begin{description}}
\newcommand{\ede}{\end{description}}
\newcommand{\bit}{\begin{itemize}}
\newcommand{\eit}{\end{itemize}}
\newcommand{\noi}{\noindent}
\newcommand{\ph}{\phantom}
\usepackage{amsmath,amssymb}
\usepackage{moreverb}
\usepackage{doi}
\usepackage[version=3]{mhchem}
\newcommand{\clearemptydoublepage}{\newpage{\pagestyle{empty}\cleardoublepage}}
\parindent 0cm

\newcommand{\llarge}[1]{\Large{#1}}
\newcommand{\lnormal}[1]{\normalsize{#1}}
\newcommand{\lsmall}[1]{\small{#1}}
\newcommand{\lfoot}[1]{\footnotesize{#1}}
\newcommand{\ft}[1]{\texttt{#1}}
\newcommand{\fb}[1]{\textbf{#1}}

\begin{document}

\thispagestyle{empty}

\vspace*{4cm}
\begin{center}

{\LARGE\bf 2-Dimensional Finite Difference \\[10pt]
Hartree-Fock Program}

\vskip 0.5cm

{\large with Libxc, OpenMP and p-thread support }

\vskip 1.0cm

{\Large User's Guide}

\vskip 1.0cm

{\large  © 1996-2024 Jacek Kobus \\[10pt]}

{ July 2024\\[10pt]}

\end{center}

\clearemptydoublepage
\tableofcontents
\thispagestyle{empty}
\clearemptydoublepage
\section{Description of input data}

x2dhf program accepts input data that consist of separate lines that contain
\begin{itemize}
\item[--] a label
\item[--] a label followed by a string of characters, integer(s)
  and/or real number(s)
\item[--] a string of characters, integer(s) and/or real number(s)
\end{itemize}
Real numbers can be written in a fixed-point or scientific notation.
\noindent
Note that
\begin{itemize}
\item[--] labels and strings can be in upper or lower case,

\item[--] the compulsory labels must follow the order given below; the
  optional ones can be~inserted anywhere between the \textbf{title}
  and \textbf{stop} labels,

\item[--] optional parameters are enclosed in square brackets,

\item[--] $r$ denotes a real number, $i$ -- an integer, $c$ -- a string
          of characters,

\item[--] an exclamation mark or a hash placed anywhere in an input
  line starts a comment and what follows ``!'' or ``\#'' is ignored.

\end{itemize}

\noindent
\subsection{Mandatory labels}

The following labels must be specified in the specified order:
\begin{description}
\item \textbf{TITLE}
\begin{description}
\item[Format:] \textbf{title} $c$ \\ $c$ is any string of up to 74
  characters describing the current case. This string is added as a
  header to a text file with a .dat extension that contains basic data
  identifying a given case, i.e. atomic numbers of nuclei, grid size
  and the number of electrons and orbital and exchange functions.
\end{description}

\item \textbf{NUCLEI}
\begin{description}
\item[Format:] \textbf{nuclei} $Z_A \;\;Z_B\;\; R \;\;[\; c\;]$ \\
  Set the nuclei charges and the bond length.

\begin{itemize}
\item[$Z_A$:] nuclear charge of centre A (real)
\item[$Z_B$:] nuclear charge of centre B (real)
\item[$R$:] bond length (real)
\item[$c$:] $angstrom$ -- the internuclear separation can be given in
  \AA{}ngstr\"om if this string is included (the conversion factor
  0.529177249 is used)
\end{itemize}
  If $|Z_A-Z_B|<10^{-6}$ then the molecule is considered to be a
  homonuclear one (see \ft{initVariables}).
\end{description}

\item \textbf{CONFIG}
\begin{description}
\item[Format:] \textbf{config} $i$
\begin{itemize}
\item[$i$:] the total charge of a system
\end{itemize}
\end{description}

The following cards define molecular orbitals and their occupation
{\bf in inverse order - the highest orbitals are on top and the lowest
  on bottom}. Note that the last orbital description card
(corresponding to the deepest lying orbital) must contain the $end$
label.

The possible formats are:
\begin{description}
\item[Format:] $i\;\;\;c$
\begin{itemize}
\item[$i$:] number of fully occupied orbitals of a given irreducible
     representation (irrep) of the $C_{\infty\,v}$ group;
	two electrons make $\sigma$~orbitals fully occupied
     and four electrons -- orbitals of other symmetries
\item[$c$:] symbol of the $C_{\infty\,v}$ irrep to which the orbitals
     belong ({\it sigma, pi, delta} or~{\it phi})
\end{itemize}
\end{description}


\begin{description}
\item[Format:] $i\;\;\;c_1 \;\;c_2$
\begin{itemize}
\item[$i$:] number of fully occupied orbitals of a given irrep of the
$D_{\infty\,h}$ group
\item[$c_1$:] symbol of the $C_{\infty\,v}$ irrep to which the orbitals belong
       ({\it sigma, pi, delta} or~{\it phi})
\item[$c_2$:] symbol for the inversion symmetry of the $D_{\infty\,h}$ irreps
       ({\it u} or {\it g})
\end{itemize}
Use this format for a homonuclear molecule unless \textbf{break} card
is~included.
\end{description}


\begin{description}
\item[Format:]
$i\;\;\;c_1\;\;c_2\;\;[c_3\;\;[\;c_4\;\;[\;c_5\;]\;]\;]$
\begin{itemize}
\item[$i$:] number of orbitals of a given irrep of the $C_{\infty\,v}$  group
\item[$c_1$:] symbol for the $C_{\infty\,v}$ irreps to which the
orbitals belong ({\it sigma, pi, delta, phi})
\item[$c_2$-$c_5$:] $+,\;-$ or . (a dot); $+/-$ denotes spin up/down
electron and . denotes an unoccupied spin-orbital
\end{itemize}
\end{description}

\begin{description}
\item[Format:] $i\;\;\;c_1\;\;c_2\;\;c_3\;\;[\;c_4\;\;[\;c_5\;\;[\;c_6\;]\;]\;]$
\begin{itemize}
\item[$i$:]  number of orbitals of a given irrep of the
$D_{\infty\,h}$ group
\item[$c_1$:] symbol for the $C_{\infty\,v}$ irrep to which the
orbitals belong ({\it sigma, pi, delta, phi})
\item[$c_2$:] symbol for the inversion symmetry of the $D_{\infty\,h}$
irrep ({\it u} or {\it g})
\item[$c_3$-$c_6$:] $+,\;-$ or . (a dot); $+/-$ denotes spin up/down
electron and . denotes an unoccupied spin-orbital
\end{itemize}
\end{description}

\item \textbf{GRID}

\begin{description}
\item[Format:] \textbf{grid} $N_{\nu}$ $R_{\infty}$ \\
An integer and a real define a single two-dimensional grid.
\begin{itemize}
\item[$N_{\nu}$:] the number of grid points in $\nu$ variable
\item[$R_{\infty}$:] the practical infinity
\end{itemize}
$N_{\mu}$ is calculated so as to make the step size in $\mu$ variable equal
to the step size in $\nu$ variable. $N_{\nu}$ and $N_{\mu}$ have to~meet
special conditions due to the order of discretization used and
the multi-threaded version of the MCSOR routine. If the conditions are not
fulfilled the nearest (but smaller) appropriate values are used.
\end{description}

\begin{description}
\item[Format:] \textbf{grid} $N_{\nu}$ $N_{\mu}$ $R_{\infty}$ \\
Two integers and one real define a single two-dimensional grid.
\begin{itemize}
\item[$N_{\nu}$:] the number of grid points in $\nu$ variable
\item[$N_{\mu}$:] the number of grid points in $\mu$ variable
\item[$R_{\infty}$:] the practical infinity
\end{itemize}
This format may be needed when interpolation between grids is attempted.
\end{description}

\item \textbf{ORBPOT}
\begin{description}
\item[Format:] \textbf{orbpot} $c$ $[\; i \;]$\\
  where $c$ a character string determining the initial source of orbitals
  and potentials.

  The following values of $c$ are allowed:
\begin{itemize}

\item $gauss$ -- GAUSSIAN output is used to retrieve the basis set and the
  molecular orbital expansion coefficients. The GAUSSIAN output is assumed
  to be contained in gaussian.out and gaussian.pun files.\footnote{The
    program has been tested on output files produced by the 94, 03, 09 and
    16 versions of the GAUSSIAN system of programs.} Coulomb and exchange
  potentials are initialized as in the \textsl{hydrogen} case; see
  \ft{prepGauss} for more details. To generate the necessary output from
  GAUSSIAN, you need to specify {\bf punch=mo gfinput} in the route section
  of the GAUSSIAN input file. When you've run GAUSSIAN, you should end up
  with a file {\bf fort.7} that contains the molecular orbitals. To run
  {\bf x2dhf}, put the log file in the working directory as {\bf
    gaussian.out} and the molecular orbitals from {\bf fort.7} as {\bf
    gaussian.pun}, which will be parsed by the program.

\item $hf$ -- radial Hartree--Fock orbitals for the centre A and B obtained
  from the qrhf program are retrieved from disk (see
  \$X2DHF\_DI\-REC\-TO\-RY/\-hf\_orbitals directory) and Coulomb and
  exchange potentials are initialized unless the OED method is chosen (see
  routine \ft{initHF} for details).  Note also that the environment
  variable X2DHF\_DIRECTORY must be set as the root of the x2dhf
  installation. If you don't set X2DHF\_DIRECTORY, the program will try to
  read the orbitals from the current working directory.
  
\item $hydrogen$ -- molecular orbitals are formed as a linear combination
  of hydrogenic functions on centres $A$ and $B$ as defined via the
  \textbf{lcao} label. In the case of HF or HFS calculations Coulomb
  (exchange) potentials are approximated as a~linear combination of
  Thomas-Fermi ($1/r$) potentials at the two centres. If the OED method
  is~chosen the potential function is~approximated as a linear combination
  of $Z_A/r_1$ and $Z_B/r_2$ terms and the exchange potentials are set to
  zero.

\item $lda$ -- radial LDA orbitals for the centre A and B obtained from the
  \textsc{HelFEM} program%
  \footnote{S.Lehtola, \textsl{Fully numerical Hartree-Fock and density
      functional calculations. I. Atoms}, Int. J. Quantum
    Chem. \doi{10.1002/qua.25945}} are retrieved from disk (see
  \$X2DHF\_DI\-REC\-TO\-RY/\-lda\_orbitals directory) and Coulomb and
  exchange potentials are initialized unless the OED method is chosen (the
  Thomas-Fermi potential is used to initialise Coulomb potentials). Note
  also that the environment variable X2DHF\_DIRECTORY must be set as the
  root of the x2dhf installation. If you don't set X2DHF\_DIRECTORY, the
  program will try to read the orbitals from the current working directory.

\item $ldasap$ -- radial LDA orbitals for the centre A and B obtained from
  the \textsc{HelFEM} program (see the previous item) but the Coulomb
  potentials are initialised using the effective Coulomb charges (see the
  \ft{effective\_coulomb\_charge} function).
  
\item $noexch$ -- orbitals and Coulomb potentials are retrieved from disk
  files and exchange potentials are initialized as in $hydrogen$ case; this
  is useful when going from DFT/HFS to HF calculations.

\item $nodat$ -- initial orbitals and potentials are retrieved from disk
  files but the content of a 2dhf\_input.dat file is retrieved from a
  2dhf\_input.orb (binary) file. Use this value when reading binary data
  generated by the earlier than 2.0 versions of the program.

\item $old$ -- initial orbitals, Coulomb, and exchange potentials are
  retrieved from disk files (2dhf\_\-input.\-orb, 2dhf\_\-input.\-coul and
  2dhf\_input.exch, respectively) created in a previous run. Data defining
  the case are retrieved from a 2dhf\_input.\-dat textfile.


\item $qrhf$ -- radial Hartree--Fock orbitals for the centre A and B
  obtained from the qrhf program\footnote{J. Kobus and Ch. Froese Fischer,
    \textsl{Quasi-Relativistic Hartree--Fock program for Atoms}, to be
    published.}  are retrieved from disk files 1dhf\-\_centreA.\-orb and
  1dhf\-\_centreB.\-orb, respectively, and Coulomb and exchange potentials
  are initialized as in the \textsl{hydrogen} case (see routine \ft{initHF}
  for details).
\end{itemize}

$i$ is an optional parameter that is used only for cases other than
$old$. To perform SCF iterations in a stable manner, when both orbitals and
potentials are being relaxed, the potentials should correspond to the
orbitals being relaxed. However, this is not the case at the very beginning
of the SCF process. Therefore, at the onset of the SCF iterations with the
fairly good initial orbitals, the orbitals can be kept fixed and only
Coulomb and exchange potentials should undergo relaxation. These initial
relaxations of the potentials are performed until the maximum error in the
orbital energy is lower than $10^{-i}$ for \textsl{nscfextra} iterations
(by default \textsl{nscfextra=3}, see \ft{initVariables}). If $i=0$ (the
default value) then both the orbitals and potentials are relaxed at every
SCF iteration.
\end{description}

\item \textbf{STOP}
\begin{description}
\item[Format:] \textbf{stop}\\
This label indicates the end of input data.
\end{description}

\end{description}

\subsection{Optional labels}

The following additional labels can be specified in any order:
\begin{description}

\item \textbf{ALTSWEEP[S]}
\begin{description}
\item[Format:] \textbf{altsweep} or \textbf{altsweeps}\\
  In the case of near-degenerate orbitals, i.e. when performing
  calculations for homonuclear molecules without inforced symmetry (no
  ``HOMO'' label), especially when the external electric field is applied,
  one can improve convergence by changing the direction of the SOR sweeps,
  i.e. using forward/backward sweeps for even/odd SCF iterations.
\end{description}

  
\item \textbf{BREAK}
\begin{description}
\item[Format:] \textbf{break} \\ When this label is present homonuclear
  molecules are calculated in $C_{\infty\,v}$ symmetry and the
  $D_{\infty\,h}$ symmetry labels ({\it u} or {\it g}) are redundant.

\end{description}


\item \textbf{CONV}
\begin{description}
\item[Format:] \textbf{conv} $[\;i_1\;[\;i_2\;[\;i_3\;]\;]\;]$\\ Sometimes the requested
  accuracy of a solution is set too high and cannot be satisfied on a~selected grid. As
  a~result SCF/SOR iteration process may continue in vain. To~save CPU time the iterations
  are stopped if orbital energies or orbital norms display no~improvement over the $i_2$
  and $i_3$ most recent iterations, respectively (20 by default). This mechanism is
  activated after $i_1$~initial iterations (600 by default).
\end{description}

\item \textbf{DEBUG}
\begin{description}
\item[Format:] \textbf{debug} $[\; i_1 \; [\;i_2 \ldots \;] \;i_{40}\;]$ \\
  Up to 40 different debug flags can be set at a time. If the integer
  $i_k$ is present the debug flag $i_k$ is set, i.e. idbg$(i_k)=1$
  $(1 \leq i_k<999)$. These are used to generate additional debugging
  information by adding the lines of the form
  \begin{verbatim}
       if (idbg(ik).eq.1) then
           print *, ``debugging something ...''
           ...
      endif
    \end{verbatim}
%These are used to generate debug information.
%If a debug flag is set to 500 an additional card is read in with
%$incrni$ and $incrmu$ variables which are
%parameters of PMTX routine used to print matrices (every $incrni$ row
%and every $incrmu$ column is printed.
\end{description}

\item \textbf{DFT}
\begin{description}
\item[Format:] \textbf{dft}  $[\;c_1\;]\;\; [\;c_2\;]$ \\
\begin{itemize}
\item[$c_i:$] specifies the type of DFT exchange-correlation functional(s) to be used in
  Fock equations. Any of the suitable LDA, GGA, and hybrid functionals implemented in Libxc
  library%
  \footnote{Susi Lehtola, Conrad Steigemann, Micael J.T. Oliveira, and Miguel
    A.L. Marques, \textsl{Recent developments in Libxc - A comprehensive library of
      functionals for density functional theory}, Software X 7, 1 (2018)} can in principle
  be used, e.g. xc\_lda\_x, xc\_lda\_c\_wigner, xc\_lda\_c\_xalpha, xc\_gga\_c\_gam, etc;
  use \texttt{lxcctl list} utility to generate the list of available functionals (the
  utility can also be used to test the functionals, see \texttt{lxcctl help}).

  \bigskip
  
  For backward compatibility the following functionals can also be selected:
  \begin{itemize}

\item $lda$ -- the local density approximation with the potential
\begin{equation*}
         V_X(\alpha)=-\frac{3}{2} \alpha \left(\frac{3}{\pi}\right)^{1/3} 2^{-2/3}
                \sum_{\sigma} \rho_{\sigma}^{1/3}
\end{equation*}
where $\alpha$ is by default set to 2/3 (the Slater exchange
potential). \linebreak[4] To change this value use the \textbf{xalpha} label.
This functional is equivalent to \textsl{lda\_x} one of the Libxc library.

If \textsl{xalpha} parameter is set to 1/3 then \textsl{lda} functional is equivalent to
\textsl{xc\_lda\_c\_xalpha} one.

\item $b88$ -- the Becke exchange potential; this functional is equivalent to
  \textsl{xc\_gga\_x\_b88} one from the Libxc library%
  \footnote{Cf. the results of test sets he/set-09, he/set-10 and he/set-11.}

\item $lyp$ -- the correlation potential of Lee, Yang and Parr; this
  functional is equivalent to \textsl{xc\_gga\_c\_lyp} one from the Libxc
  library

\item $vwn$ -- the correlation potential of Vosko, Wilk and Nusair;
  this functional is equivalent to \textsl{lda\_c\_vwn} one from the Libxc library%
    \footnote{In case the Hartree-Fock orbitals of the FH molecule are used to calculate
      the correlation contribution the functional \textsl{vwn} gives -0.699579 while
      \textsl{xc\_lda\_c\_vwn} gives a slightly smaller value, namely, -0.704211.}

\end{itemize}
\end{itemize}
\end{description}
When the bare label is present and the method selected is HF then the exchange
contributions (LDA, B88) and the correlation contributions (LYP and VWN) to the total
energy are calculated upon completion of the SCF iterations.

\item \textbf{DENSTHLD}
\begin{description}
\item[Format:] \textbf{densthld}  $[\;r\;]$ \\
\begin{itemize}
\item[$r:$] sets the value of \texttt{dens\_threshold} parameter used in the
  Libxc library to avoid returning NaN values (the default value is defined in
  \ft{initVariables} subroutine.%
\end{itemize}
\end{description}

\item \textbf{FASTSCF}
\begin{description}
\item[Format:] \textbf{fastscf} \\
  This label causes the SCF process to automatically skip relaxing orbitals
  (and also Coulomb and exchange potentials) for which the orbital energy
  threshold has been reached. This is a very desirable behaviour when
  (and the default one) when the OED method is used to calculate a set of
  virtual orbitals.
   \end{description}

\item \textbf{FEFIELD}
\begin{description}
\item[Format:] \textbf{fefield} $r$
\begin{itemize}
\item[$r$:] a strength of an external static electric field directed along
  the internuclear axis (in atomic units)
\end{itemize}
\end{description}



\item \textbf{FERMI}
\begin{description}
\item[Format:] \textbf{fermi} [$r_A$ [ $r_B$ ]] \\ When this label is
  present the Fermi nuclear charge distribution is used. Optional
  parameters $r_A$ and $r_B$ define the atomic masses (in atomic mass
  units, amu) of nuclei A and B. If omitted the corresponding values are
  taken from the table of atomic masses compiled by Wapstra and Audi (see
  \ft{blk\_data}).
\end{description}


\item \textbf{FIXORB}
\begin{description}
\item[Format:] \textbf{fixorb} $[\; i_1 \; [\;i_2 \ldots \;] \;i_{40}\;]$
  \\ This label is used to specify orbitals to be kept frozen during
  SCF/SOR process (they are not being renormalized nor orthogonalized
  during the process). $i_1$, $i_2$, $\ldots$ are the numbers of these
  orbitals as they appear on the program's listing, i.e. their order is
  reversed to that used when defining the electronic configuration (see the
  \fb{config} card). Up to 40 different orbitals can be set at a~time. Use
  the bare label to keep all orbitals frozen.

\end{description}

\item \textbf{FIXPOT}
\begin{description}
\item[Format:] \textbf{fixpot}\\ If this label is present then all
  Coulomb and exchange potentials are kept frozen during the SCF/SOR process.
\end{description}

% \item \textbf{FIXEXCH}
% \begin{description}
% \item[Format:] \textbf{fixexch}\\ If this label is present then all
%   exchange potentials are kept frozen during the SCF/SOR process.
% \end{description}


\item \textbf{GAUSS}
\begin{description}
\item[Format:] \textbf{gauss} [$r_A$ [$r_B$]] \\ When this label is present
  the Gauss nuclear charge distribution is used. Optional parameters $r_A$
  and $r_B$ define the atomic masses (in amu) of nuclei A and B. If
  omitted, the corresponding values are taken from the table of atomic
  masses compiled by Wapstra and Audi (see \ft{blk\_data}).
\end{description}

\item \textbf{HOMO}
\begin{description}
\item[Format:] \textbf{homo}  \\
  This label is used to impose explicitly $D_{\infty\,h}$ symmetry upon
  orbitals of homonuclear molecules to improve SCF/SOR convergence.
\end{description}

\item \textbf{INOUT}
\begin{description}
\item[Format:] \textbf{inout} $c_1$ $c_2$\\ The x2dhf program can be
  compiled to support calculation using three different combinations of
  integer/real data types: i32 (4-byte integers, 8-byte reals), i64 (8-byte
  integers, 8-byte reals) and r128 (8-byte integers, 16-byte reals).
  Strings $c_1$ and $c_2$ determine the combination appropriate for the
  format of~input and output data, respectively, and each string can be
  i32, i64 or r128.

  To facilitate the exchange of binary data generated on machines
  of~different architectures or using different compilers additional
  formats are available, namely i32f, i64f or r128f. They can be used to
  export/import data in the formatted instead of the (default) unformatted form.
\end{description}


\item \textbf{INTERP}
\begin{description}
\item[Format:] \textbf{interp} $[\; {\rm iord}_\nu^{\rm orb}$ ${\rm
    iord}_\mu^{\rm orb}$ ${\rm iord}_\nu^{\rm coul}$ ${\rm
    iord}_\mu^{\rm coul}$ ${\rm iord}_\nu^{\rm exch}$ ${\rm
    iord}_\mu^{\rm exch} ]$\\ Use this label to~change the grid
  between separate runs of the program. The interpolation order in the
  $\nu$ and $\mu$ directions used for the orbitals, Coulomb potential
  and exchange potentials can be set with the six optional
  parameters. Supported orders for the interpolation are 2, 4
  (default), 6, and 8.
\end{description}

\item \textbf{KINPOT}
\begin{description}
\item[Format:] \textbf{kinpot}  \\
  Upon the completion of the SCF process the Pauli, $v^P(\fb{r})$, and
  von Weizs\"acker kinetic potentials, $v^W(\fb{r})$, are calculated
  along the internuclear axis and written to out4kinpot.dat file together
  with the total density, $\rho(\fb{r})$, $\nabla^2 \rho(\fb{r})$,
  $\tau(\fb{r})$ (the kinetic energy density) and 
  $\tau^W(\fb{r})=|\nabla \rho(\fb{r})|^2/(8\rho(\fb{r}))$,
\end{description}

\item \textbf{LCAO}
\begin{description}
\item[Format:] \textbf{lcao} $[\;i\;]$\\ If the source of orbitals is
  declared as \textsl{hydrogen} then this card must be present. In such a
  case the initialization of each of the orbitals has to be defined
  in~terms of the linear combination of atom centred hydrogen-like
  functions. For each orbital include a card of the following format (make
  sure that the order of orbitals should match the order specified under
  the \textbf{config} label):

\begin{description}
\item[Format:] $c_A\;\;n_A \;\;l_A \;\;\zeta_A \;\;\;\;c_B\;\;n_B\;\;l_B\;\;
  \zeta_B\;\;\;[\;i_1\;\;\;\;[\;i_2\;]\;]$ \\
where
\begin{itemize}
\item[] $c_A$ -- relative mixing coefficient for a hydrogenic orbital on the
                    $Z_A$ centre (real),
\item[] $n_A$ -- its principal quantum number (integer)
\item[] $l_A$ -- its orbital quantum number (integer)
\item[] $\zeta_A$ -- the effective nuclear charge if $i=1$ (default) or
a screening parameter if $i=2$ (real)
\item[] $c_B$  -- relative mixing coefficient for a hydrogenic orbital on the
                    $Z_B$ centre (real),
\item[] $n_B$ -- its principal quantum number (integer)
\item[] $l_B$ -- its orbital quantum number (integer)
\item[] $\zeta_B$ -- the effective nuclear charge if $i=1$ (default) or
a screening parameter if $i=2$ (real)
\item[] $i_1$ -- (integer) if set to 1 (the default) indicates that this
  orbital should be initialized as a linear combination of hydrogenic
  functions and not taken from a disk orbital file (integer).
  When the source of orbitals is declared as \textsl{old} and the
  orbital and Coulomb potential data files contain fewer functions than
  defined by \textsl{config} card this flag can be used to indicate which
  orbitals are missing in orbital data files and require
  initialization. This can be useful when generating virtual orbitals for a
  potential formed from the already given set of converged orbitals. Indicate
  such orbitals by setting this parameter to 0.

\item[] $i_2$ -- (integer) if set to 0 (1, the default) indicates that this
  orbital will not contribute (will contribute) to the electron density
  when evaluating DFT functionals. It allows therefore to generate virtual
  orbitals within the DFT calculations since a local potential can be built
  from a given subset of orbitals that are kept frozen during the SCF
  process (see \fb{fixorb} label).
\end{itemize}
The mixing coefficients are normalized so that $|c_A|+|c_B|=1$

\end{description}
\end{description}

\item \textbf{LM}
\begin{description}
\item[Format:] \textbf{lm} $i_1 \; i_2$ [\;$i_3 \;i_4$\;] \ldots [\;$i_{N-1} \;i_{N}$\;] \\
  For closed-shell systems, the off-diagonal Lagrange multipliers are set to
  zero. However, for some such systems the SCF/SOR convergence can be
  improved if for some pair(s) of orbitals the non-zero values of the
  Lagrange multipliers are used (see, for example, the o2, li2-elprop or
  f2-elprop test suits).
\end{description}

\item \textbf{MCSOR$|$MCSOR-O}
\begin{description}
\item[Format:] \textbf{mcsor|mcsor-o} $[\;i_1\;[\;i_2\;[\;i_3\;]\;]$\\
  Selects the MCSOR method for solving the Poisson equations for orbitals
  and the SOR method -- for Coulomb and exchange potentials. $i_1$
  parameter can be used to set the number of threads employed. The default
  value depends on the parallelization method used; see
  \ft{initVariables}. Use $i_2$ and $i_3$ to set the value of the 
  macro and micro MCSOR relaxation sweeps, respectively, during a single SCF
  cycle for orbitals and potentials, respectively (by default
  $i_2=1, i_3=10$).

  When the MCSOR method is selected via this label then the single-threaded
  version of the multi-colour SOR method is chosen by default to relax
  orbitals. When OPENMP/PTHREAD/TPOOL directive is present the parallelized
  version of MCSOR is employed.%
  \footnote{The TPOOL directive should be used together with the MUTEX one
    (see \ft{x2dhfctl -t -X}.}  Coulomb and exchange potentials are relaxed
  in parallel using the single-threaded SOR method. However, when the
  MCSOR-CE label together with OPENMP directive is used, also the
  potentials are relaxed using the parallelized version of the MCSOR
  routine (in the case of PTHREAD/TPOOL directive is used the SOR method
  must be used to avoid nested parallel regions).

  Labels MCSOR-O and MCSOR-CE were added to force the usage
  of the MCSOR selectively for orbitals and Coulomb/exchange potentials,
  respectively. See \ft{relaxDriver} for more details.
  
\end{description}

\item \textbf{MCSOR-O}
\begin{description}
\item[Format:] \textbf{mcsor-o} $[\;i_1\;[\;i_2\;[\;i_3\;]\;]$\\
  Selects the MCSOR method for solving the Poisson equations for orbitals
  and the SOR method -- for Coulomb and exchange potentials. $i_1$
  parameter can be used to set the number of threads employed. The default
  value depends on the parallelization method used; see
  \ft{initVariables}. Use $i_2$ and $i_3$ to set the value of the micro and
  macro MCSOR relaxation sweeps, respectively, during a single SCF cycle
  for orbitals (by default $i_2=10, i_3=1$).
\end{description}

\item \textbf{MCSOR-CE}
\begin{description}
\item[Format:] \textbf{mcsor-ce} $[\;i_1\;[\;i_2\;[\;i_3\;]\;]$\\
  When OPENMP directive is used this label selects the MCSOR method for
  solving the Poisson equations for Coulomb and exchange potentials.  $i_1$
  parameter can be used to~set the number of threads employed. Use $i_2$
  and $i_3$ to set the value of the micro and macro MCSOR relaxation
  sweeps, respectively, during a single SCF cycle for potentials (by
  default $i_2=10, i_3=1$). In case of PTHREAD/TPOOL directive the SOR
  method is used (see \ft{coulExchSORPT}).
\end{description}

\item \textbf{METHOD}
\begin{description}
\item[Format:] \textbf{method} $c$\\
Select the type of calculation.
\begin{itemize}
\item[$c$:] HF -- the Hartree-Fock method
\item[$c$:] DFT -- the Hartree-Fock method with the $X\alpha$ exchange
  potential ($\alpha=2/3$); see the \fb{dft} label to~choose another
  exchange or correlation potential

\item[$c$:] HFS -- the Hartree-Fock-Slater method (Hartree-Fock with the
  $X\alpha$ exchange potential) with the optimum value of the $\alpha$
  parameter (see \ft{src/commons.F90})

\item[$c$:] OED -- One Electron Diatomic ground and excited states can be
  calculated for the Coulomb potential in the prolate spheroidal
  coordinates (default). It is also possible to specify the Coulomb and
  Kramers-Henneberger potentials in cylindrical coordinates (see the
  \fb{poth3}, \fb{potkh}, \fb{potharm2}, \fb{potharm3} labels,
  respectively). When more than one orbital is specified calculations are
  carried out as if in the case of a multielectron system.\footnote{In this
    type of calculations convergence rates differ greatly between
    orbitals. Therefore, if for a given orbital the orbital energy
    threshold is reached it is being frozen.}

\item[$c$:] TED -- Two Electron Diatomic ground and excited states can be
  calculated for Hooke's atom (harmonium). See \textbf{POTHOOKE} label.

\item[$c$:] SCMC -- the Hartree--Fock method with $X\alpha$ exchange where
  the $\alpha$ parameter is calculated according to the self-consistent
  multiplicative constant method\footnote{V. V. Karasiev and
    E. V. Lude\~{n}a, \textsl{Self-consistent multiplicative constant
      method for the exchange energy in density functional theory},
    Phys. Rev. A~\textbf{65} (2002)
    062510. \doi{10.1103/PhysRevA.65.062510}}

\end{itemize}
\end{description}

\item \textbf{MMOMENTS}
\begin{description}
\item[Format:] \textbf{mmoments} \\
  Calculate the multipole moments for the converged orbitals relative to
  the geometric centre and the centre of mass. The total charge densities at
  (0,0,-R/2) and (0,0,+R/2) are also evaluated.
\end{description}

\item \textbf{MULTIPOL}
\begin{description}
\item[Format:] \textbf{multipol} $r$ [ $i$ ]
\begin{itemize}
\item[$r$:] if $r>0$ multipole moment expansion coefficients are
  recalculated when the maximum error in orbital energy is changed by $r$
  (the default value is 1.15; see \ft{initVariables}). To suppress
  recalculation of the coefficients set $r$ to a negative real number. This
  is useful when generating potentials from a set of fixed orbitals,
  e.g. from GAUSSIAN orbitals.

\item[$i$:] the number of terms in the multipole expansion used to calculate
  the boundary values for potentials ($2 \le i\le 8$ and the default is 4).
\end{itemize}
\end{description}

\item \textbf{OUT4DD}
\begin{description}
\item[Format:] \textbf{out4dd}  \\ 
  Write orbitals in formatted form to out4dd.dat file (to be subsequently
  used by the x5dhf program).
\end{description}


\item \textbf{OMEGA}
\begin{description}
\item[Format:] \textbf{omega} $\omega_{orb}$ [ $\omega_{pot}$ ] \\ 
  One or two real numbers setting over-relaxation parameters for relaxation
  of orbitals and potentials; see also \textbf{OMEGAOPT}. The negative
  value of $\omega_{orb}$/$\omega_{pot}$ indicates that its value should be
  set to a~near optimum value obtained from a~semiempirical formula (see
  \ft{initCBlocks} and \ft{setOmega} for details).
\end{description}

\item \textbf{OMEGAOPT}
\begin{description}
\item[Format:] \textbf{omegaopt} [ i [ $r_{orb}$ [ $r_{pot}$ ] ] ] \\
  The optional integer parameter can be set to 1 (default) or 2. In the
  former case the over-relaxation parameters are based on tests performed
  for the FH molecule on 157x229/45 grid and scaled down a bit (by 0.979
  and 996 for orbitals and potentials, respectively). In the latter, the
  values are calculated from a formula for the optimal value of the
  over-relaxation parameter in case the model Poisson equation is
  discretized using the second-order stencil.

  \textbf{This label should only be used for test purposes.}
\end{description}

where $c$ a character string determining the initial source of orbitals

\item \textbf{ORDER}
\begin{description}
\item[Format:] \textbf{order} $c$\\
  where $c$ is a character string defining the ordering of mesh points:
  ``col-wise'' -- natural column-wise, ``middle'' -- 'middle' type of sweep
  (default), ``row-wise'' -- natural row-wise, ``rcol-wise'' -- reversed
  natural column-wise (see \ft{mesh} routine for details)
\end{description}

\item \textbf{PLOT}
\begin{description}
\item[Format:] \textbf{plot} $[\;i_1\;[\;i_2\;]\;]$\\
  Orbital values are written in formatted form to be plotted.  For $i_1$
  equal to 1, 2 or 3 values $(z_j,x_k,f(z_j,x_k))$,
  $(\nu_j,\mu_k,f(\nu_j,\mu_k))$ or $(\eta_j,\xi_k,f(\eta_j,\xi_k))$ are
  written, respectively, and every $i_2^{\rm th}$ value in each variable is
  selected, i.e. $j=1,N_{\nu},i_2$ and $k=1,N_{\mu},i_2$. By default $i_1=1$
  and $i_2=2$. The values in the tail region are omitted if they are
  smaller that a given threshold (by default $10^{-10}$; see
  \texttt{plotThreshold} variable in \texttt{initVariables} routine).

\end{description}


\item \textbf{POTGSZ}
\begin{description}
\item[Format:] \textbf{potgsz} \\ When the OED method is chosen then
  this label selects a model potential due to Green, Sellin and
  Zachor.\footnote{A. E. S. Green, D. L. Sellin, and A. S. Zachor,
    \textsl{Analytic Independent-Particle Model for Atoms},
    Phys. Rev. \textbf{184} (1969) 1-9. \doi{10.1103/PhysRev.184.1}}
  For a given atom this potential produces HF-like orbitals and it has
  been found useful in finding decent starting orbitals for diatomic
  molecules as well.\footnote{J. E. Whalen and A. E. S. Green,
    \textsl{Analytic Independent Particle Model for Molecules},
    Am. J. Phys. \textbf{40}, (1972) 1484. \doi{10.1119/1.1986874};
    K. J. Miller and A. E. S. Green, \textsl{Energy levels and
      potential energy curves for H2, N2, and O2 with an independent
      particle model}, J. Chem. Phys. \textbf{60}, (1974)
    2617. \doi{10.1063/1.1681415}; T. Sawada, P. S. Ganas, and
    A. E. S. Green, \textsl{Elastic scattering of electrons from
      \ce{N2}}, Phys. Rev. A \textbf{9}, (1974)
    1130. \doi{10.1103/PhysRevA.9.1130}} Note, however, that the
  superposition of atomic potentials (SAP) guess\footnote{S. Lehtola,
    \textsl{Assessment of Initial Guesses for Self-Consistent Field
      Calculations. Superposition of Atomic Potentials: Simple yet
      Efficient}, J. Chem. Theory Comput. \textbf{15}, (2019)
    1593-1604. \doi{10.1021/acs.jctc.8b01089}} (keyword:
  \textbf{potsap}) is much more accurate, as it is not restricted to a
  simple analytic form.
\end{description}

\item \textbf{POTGSZG}
\begin{description}
\item[Format:] \textbf{potgszg} \\ When the OED method is chosen then
  this label selects a~model potential due to Green, Sellin and Zachor
  and the Gauss nuclear charge distribution. For a~given atom this
  potential produces HF-like orbitals but it has been found useful in
  finding decent starting orbitals for any molecular system.
\end{description}

\item \textbf{POTHOOKE}
\begin{description}
\item[Format:] \textbf{pothooke} $k$ \\
  When the HF method is chosen this label selects the
  potential of the Hooke atom: $-\frac{1}{2}k^2( r_1^2 + r_2^2)$ ($k=1/2$
  by default), where $r_1$ and $r_2$ denote the distances of the two
  electrons from the center A and thus the nuclear charges should be set
  accordingly, e.g. $Z_1=2$ and $Z_2=0$. 
\end{description}

\item \textbf{POTHARM2}
\begin{description}
\item[Format:] \textbf{potharm2} $k$ \\ When the OED method is chosen then
  this label selects a~two-dimensional model potential of the form
  $-k^2(x^2+y^2)$, where $k$ is a strength of the potential (real).%
\end{description}

\item \textbf{POTHARM3$|$EXTRACUL[E]}
\begin{description}
\item[Format:] \textbf{potharm3$|$extracule[e]} [$k$] \\ When the OED method
  is chosen then this label selects a~three-dimensional model potential of
  the form $-k^2(x^2+y^2+z^2)$, where $k$ is a strength of the potential
  (real, $k=1/2$ by default).
\end{description}

\item \textbf{POTCOUL2}
\begin{description}
\item[Format:] \textbf{potcoul2} $m\;\;a\;\;V_0$ \\ When the OED method
  is chosen then this label selects a~two-dimensional model potential
  of the form $ -V_0/\sqrt{a^2+x^2+y^2}$.
  The following parameters can be set
\begin{itemize}
\item [$m:$] magnetic quantum number of a state (integer)
\item [$a:$] width of the model potential (real)
\item [$V_0:$] depth of the model potential (real)
\end{itemize}
\end{description}



\item \textbf{POTCOUL3}
\begin{description}
\item[Format:] \textbf{potcoul3} $m\;\;a\;\;V_0$ \\ When the OED method
  is chosen then this label selects a~two-dimensional model potential
  of the form $ -V_0/\sqrt{a^2+x^2+y^2+z^2}$.
  % (see \ft{initCBAllocArrays.f90} for details).
  The following parameters can be set
\begin{itemize}
\item [$m:$] magnetic quantum number of a state (integer)
\item [$a:$] width of the model potential (real)
\item [$V_0:$] depth of the model potential (real)
\end{itemize}
To get the hydrogen Coulomb potential set $a=0.0$ and $V_0=1.0$. Set $a>0$
and $V>0$ to choose its smoothed variant.

\end{description}


\item \textbf{POTKH}
\begin{description}
\item[Format:] \textbf{potkh}
  $m\;\;\varepsilon\;\;\omega\;\;\;[\;a\;[\;V_0\;[\;N\;]\;]\;]$\\
 When the OED method is chosen, then this label selects the
  Kramers-Henneberger potential. The following parameters can be set
\begin{itemize}
\item [$m:$] magnetic quantum number of a state being calculated (integer)
\item [$\varepsilon:$] laser intensity (real)
\item [$\omega:$] laser cycle frequency (real)
\item [$a:$] original (before averaging over one laser cycle)
  soft-core potential width (a positive real number, by default $a=1.0$)
\item [$V_0:$] original soft-core potential depth (by default $V_0=1.0$)
\item [$N:$] number of intervals in the Simpson quadrature (an
  integer, $N=1000$ by default)
\end{itemize}
\end{description}

\item \textbf{POTSAP}
\begin{description}
\item[Format:] \textbf{potsap} \\ When the OED method is chosen, this
  label selects a pretabulated, screened atomic exchange potential
  that has been found to reproduce results in molecular
  calculations.\footnote{S. Lehtola, \textsl{Assessment of Initial
    Guesses for Self-Consistent Field Calculations. Superposition of
    Atomic Potentials: Simple yet Efficient}, J. Chem. Theory
  Comput. \textbf{15}, (2019)
  1593-1604. \doi{10.1021/acs.jctc.8b01089}} The potentials of the
  present implementation come from highly precise exchange-only LDA
  calculations carried out within the finite element formalism with
  the \textsc{HelFEM} program.\footnote{S. Lehtola, \textsl{Fully
    numerical calculations on atoms with fractional occupations and
    range-separated exchange functionals}, Phys. Rev. A 101, 012516
  (2020). \doi{10.1103/PhysRevA.101.012516}}
\end{description}

\item \textbf{PRINT}
\begin{description}
\item[Format:] \textbf{print} $[\; i_1 \; [\;i_2 \ldots \;] \;i_{40}\;]$ \\ Up to 40
  different printing flags can be set at a~time. If the integer $i_k$ is encountered the
  printing flag $i_k$ is set, i.e. iprint$(i_k)=1$ $(1 \leq i_k<999)$. These are used
  to~generate additional printouts by adding the lines of~the form
  \begin{verbatim}
       if (iprint(ik).eq.1) then
           print *, ``printing something ...''
           ...
      endif
    \end{verbatim}

Set
\begin{itemize}
\item $i_1=110$ to print a total radial density relative to the
  centre~A along the internuclear axis ($-R_{\infty}\le z\le -R/2$)

\item $i_2=111$ to print a total radial density relative to the
  centre~B along the internuclear axis ($R/2\le z\le R_{\infty}$)
\end{itemize}
See \ft{inputData} for a~list of available flags.

\end{description}


\item \textbf{PRTEVERY}
\begin{description}
\item[Format:] \textbf{prtevery} $i_1$ $i_2$ \\ Routine \ft{pmtx} can be used to output
  two-dimensional arrays in a~tabular row-wise form with every $i_1$-th row and $i_2$-th
  column being printed (by default every 10th row and column is selected)
\end{description}


\item \textbf{SCF}
\begin{description}
\item[Format:] \textbf{scf}
$[\;i_1\;[\;i_2\;[\;i_3\;[\;i_4\;[i_5]\;]\;]\;]\;]$
\begin{itemize}
\item[$i_1$:] maximum number of SCF iterations (default 1000); to skip the
  SCF step set $i_1$ to a negative integer,
\item[$i_2$:] every $i_2$ SCF iterations orbitals and potentials are saved
  to~disk (default 20).  If $i_2=0$ functions are saved on disk upon
  completion of the SCF process. If~$i_2<0$ functions are never written to
  disk,
\item[$i_3$:] if the maximum error in orbital energy is less than
  $10^{-i_3}$ for \textsl{nscfextra} iterations (3 by default; see
  \ft{initVariables}), than the SCF process is terminated ($i_3=10$ by default),
\item[$i_4$:] if the maximum error in orbital norm is less than $10^{-i_4}$
  for \textsl{nscfextra} iterations (3 by default; see \ft{initVariables}), than
  SCF process is terminated ($i_4=10$ by default),
\item[$i_5$:] the verbosity of the output during SCF process
\begin{itemize}
\item $i_5=1$ -- the orbital energy, the difference between its current and previous
  value, the normalization error, and the (absolute) value of~the largest overlap integral
  between the current orbital and all the lower-lying ones of the same symmetry (the value
  is zero for the lowest orbitals of each symmetry) is~printed for every orbital in every
  SCF iteration
\item $i_5=2$ -- the orbital energy, the difference between its current and previous value
  and the normalization error is~printed for the worst converged orbital in energy (first
  line) and norm (second line) in every SCF iteration (default)
\item $i_5=3$ -- the orbital energy, the difference between its current and previous values
  and the normalization error is~printed for the worst converged orbital in energy (first
  line) and norm (second line) every $i_2$ iterations. Printing of ``... multipole moment
  expansion coefficients (re)calculated ...''  communique is suppressed
\end{itemize}
\end{itemize}
The total energy is printed every $i_2$ iteration.
\end{description}

\item \textbf{SLOWEXCH}
\begin{description}
\item[Format:] \textbf{slowexch} \\
   This label causes the SCF process to relax the exchange potential for a given pair of
   orbitals twice during an SCF cycle.
\end{description}

\item \textbf{SOR}
\begin{description}
\item[Format:] \textbf{sor} $[\;i_1\;[\;i_2\;]\;]$\\
  Selects the SOR method for solving the Poisson equations for orbitals and
  potentials (default) and changes the value of the micro SOR relaxation
  sweeps during a single SCF cycle for orbitals ($i_1$) and potentials
  ($i_2$); by default, $i_1=i_2=10$.
\end{description}

\item \textbf{SOR4ORB}
\begin{description}
\item[Format:] \textbf{sor4orb} $[\;i_1\;[\;i_2\;]\;]$\\
  Selects the SOR method for solving the Poisson equations for orbitals and
  changes the value of the SOR micro (($i_1$) and macro ($i_2$) relaxation
  sweeps during a single SCF cycle for orbitals; by default $i_1=10, i_2=1$.
\end{description}

\item \textbf{SOR4POT}
\begin{description}
\item[Format:] \textbf{sor4pot} $[\;i_1\;[\;i_2\;]\;]$\\
  Selects the SOR method for solving the Poisson equations for potentials
  and changes the value of the SOR micro ($i_1$) and macro ($i_2$)
  relaxation sweeps during a single SCF cycle for potentials; by default
  $i_1=10, i_2=1$.
\end{description}

\item \textbf{TAIL}
\begin{description}
\item[Format:] \textbf{tail} \\ This label causes the examination of the
  tail regions of the orbitals, i.e. the regions where the $\mu$ variable
  has its maximum value, to find how fast the orbitals decay. These values
  can help choose a safe value of $R_{\infty}$.
\end{description}


\item \textbf{XALPHA}
\begin{description}
\item[Format:] \textbf{xalpha} $\alpha$ \\ This label allows to change the $\alpha$
  parameter of the LDA potential (see the HFS/DFT method); by default $\alpha$ is set
  to 2/3. To set this parameter to zero use a very small but nonzero value, e.g. \texttt{1.0e-15}.
\end{description}
\end{description}

\newpage

\section{Program's data files}

There are several standard names used by the program to keep track of its
input and output disk files. Normally the program writes out the data in
the course of computations and upon its completion into the following disk
files:
\begin{description}

\item \texttt{2dhf\_output.dat} (a text (ASCII) file) containing the title
  of a case, the time and date of its commencement, the number of mesh
  points, the internuclear distance, the charges of nuclei and the number
  of orbitals, electrons, and exchange potentials (see \texttt{writeDisk}
  for details),

\item \texttt{2dhf\_output.orb} (a binary file) containing molecular orbitals (in the
  order specified by the input data following \textbf{config} label) followed by their
  normalization factors, orbital energies, Lagrange multipliers, and multipole moment
  expansion coefficients (see \texttt{write\*}),

\item \texttt{2dhf\_output.coul} (a binary file) containing
  corresponding Coulomb potentials and

\item \texttt{2dhf\_output.exch} (a binary file) containing all exchange potentials if the
  \textbf{exchio [in-one$|$in-many] out-one} card is present

\end{description}

These files can be used to restart a given case or run another with slightly modified
parameters. If \textbf{orbpot old} card is present orbitals are retrieved from
\texttt{2dhf\-\_in\-put\-.orb} file, Coulomb potentials from \texttt{2dhf\_input.coul} and
exchange potentials from \texttt{2dhf\_input.exch} file.

\newpage

\section{Running the program}

\noindent
To simplify the usage of the program, the xhf script (see tests/xhf)
is~provided to~facilitate handling of the disk files.  The command xhf can
be invoked with one, two or three parameters. There are two basic modes of
its usage:

\begin{description}
\item \hspace*{0.5cm} \texttt{xhf file1 file2}\\ runs x2dhf reading
  input data from \texttt{file1.data} file and writing text data
  describing the case into \texttt{file2.dat} file and binary data
  with orbitals and potentials into \texttt{file2.orb},
  \texttt{file2.coul} and \texttt{file2.exch} files.
\end{description}


\begin{description}
\item \hspace*{0.5cm} \texttt{xhf file1 file2 file3}\\ runs x2dhf
  reading input data from \texttt{file1.data} and initial orbitals and
  potentials from \texttt{file2.dat}, \texttt{file2.orb},
  \texttt{file2.coul} and \texttt{file2.exch} files and writing
  resulting data into \texttt{file3.dat}, \texttt{file3.orb},
  \texttt{file3.coul} and \texttt{file3.exch} files.
\end{description}

If, for example, we have the following two data files:
\begin{itemize} 
\item \texttt{be.data}
{\fontsize{10}{10}\selectfont   
\begin{verbatim}
TITLE Be R_inf=35.0 bohr  R = 2.3860 bohr
METHOD  hf
NUCLEI  4.0  0.0  2.386  
CONFIG  0
        2 sigma  end
grid 169 35.0   
orbpot hydrogen
lcao
 1.0   2 0 4.0     0.0   1 0 9.0    
 1.0   1 0 4.0     0.0   1 0 9.0    
SCF 300 10 4 8 1
stop
\end{verbatim}
}

  
\item \texttt{be-1.data} 
{\fontsize{10}{10}\selectfont 
\begin{verbatim}
TITLE Be R_inf=35.0 bohr  R = 2.3860 bohr
METHOD  hf 
NUCLEI  4.0  0.0  2.386  
CONFIG  0
        2 sigma  end
grid 169 35.0   
orbpot old
lcao
 1.0   2 0 4.0     0.0   1 0 9.0  0  
 1.0   1 0 4.0     0.0   1 0 9.0  0  
SCF 3000 10 8 10 1
stop
\end{verbatim}
}
\end{itemize}   

then
\begin{description}
\item \hspace*{0.5cm} \texttt{xhf be be-1}
\end{description}
starts and performs the first 300 SCF iterations. And 
\begin{description}
\item \hspace*{0.5cm} \texttt{xhf be-1 be-1 be-2}
\end{description}
must be used to continue the calculations. To converge the SCF process even
better increase the convergence parameters (see the \textbf{scf} label) and
use the following command
\begin{description}
\item \hspace*{0.5cm} \texttt{xhf be-1 be-2 be-1}
\end{description}


Use \texttt{testctl list} to see dozens of examples of the x2dhf usage; try
\texttt{textctl help|-h|--help} to get more details.

\bigskip

\texttt{xhf} can be used to perform a number of
additional tasks; see \texttt{xhf help|-h|--help}.

\end{document}

\section{Useful hints}

\begin{enumerate}
\item The program should be easy to use provided you can start a
  calculation for a specific system. You should not encounter any serious
  problems when the system contains atoms from the first two rows of the
  Periodic Table. Then even the rough hydrogenic estimates of the orbitals
  should prove adequate and after an initial couple of dozen of iterations,
  a smooth convergence should set in.

  If, however, a system contains more than 15-20 electrons the initial
  estimates of the orbitals have to be good enough to avoid
  divergences. Then, you have to choose the parameters of the hydrogenic
  orbitals carefully or perform the finite basis set calculations using the
  GAUSSIAN program to provide the initialization data for orbitals.

  One can also use the HFS method to produce initial estimates of orbitals
  and Coulomb potentials. For example, to start calculations for the
  neon atom one can use the following input data:

  \verbatiminput{../examples/ne/ne-a.data}

  This input produces good enough HFS initial estimates so that the
  calculations can be continued at the HF level with the corresponding
  input:

  \verbatiminput{../examples/ne/ne-b.data}

  One can also use the HF method with some model potential, e.g. the model
  potential of Green, Sellin, Zachor (label POTGSZ) or the Superposition of
  Atomic Potentials due to Susi Lehtola (POTSAP).

\item Choose smaller values of the relaxation parameters ($1.7\leq
  \omega\leq 1.85$) to avoid divergences in the first few dozens of
  SCF iterations (values as small as 0.2 may be used for very heavy
  systems). Set the maximum number of SCF iterations to a small value
  -- between 20 and 500 -- and/or impose crude convergence criteria
  for the orbital energy and normalization.

\item Once the orbitals and potentials have reached initial
  convergence, the relaxation parameters should be increased to their
  (near) optimum value (see the \textbf{omega} label and
  Example~\ref{example-c2a}).

  It is possible to set $\omega_{pot}$ to its near-optimal value by
  calculating it from a~semi-empirical formula; see the \ft{omega}
  label. As a rule of thumb the optimal value of the orbital
  relaxation parameter is somewhat smaller and, by~default,
  is~obtained by scaling the $\omega_{pot}$ value by 0.98 (see
  \ft{initVariables}).

\item In the case of convergence problems try to perform calculations on a
  sparser grid. For example, the $[61\times 79/30]$ grid is
  sufficient to check the quality of the initial data for the Ne$_2$
  system.

\item \label{section:hints} How to stop the program gracefully during
  a lengthy calculation without killing the process and interrupting
  disk read/write operations?  All you have to do is to create a (zero
  length) file named \texttt{stop\_x2dhf} in a working directory by
  typing \texttt{xhf stop} (you can also use the Unix \texttt{touch}
  command to~this end). The program stops whenever this file is
  detected upon the completion of a~current orbital/potential
  relaxation.
\end{enumerate}
\end{description}

\end{document}
