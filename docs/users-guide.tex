%
\documentclass[12pt,a4paper]{article}
%\input ijmpc.sty
\hoffset -1.0cm
%\voffset -1.5in
\textwidth 16.0cm
\textheight 20.0cm
\newcommand{\beq}{\begin{equation}}
\newcommand{\eeq}{\end{equation}}
\newcommand{\cl} {\centerline}
\newcommand{\bde}{\begin{description}}
\newcommand{\ede}{\end{description}}
\newcommand{\bit}{\begin{itemize}}
\newcommand{\eit}{\end{itemize}}
\newcommand{\noi}{\noindent}
\newcommand{\ph}{\phantom}
%\newcommand{\date}{}
%
%\usepackage{amsfonts,amsmath,amssymb}
\usepackage{amsmath,amssymb}
\usepackage{moreverb}
\usepackage{doi}
\newcommand{\clearemptydoublepage}{\newpage{\pagestyle{empty}\cleardoublepage}}
\parindent 0cm

\newcommand{\llarge}[1]{\Large{#1}}
\newcommand{\lnormal}[1]{\normalsize{#1}}
\newcommand{\lsmall}[1]{\small{#1}}
\newcommand{\lfoot}[1]{\footnotesize{#1}}
\newcommand{\ft}[1]{\texttt{#1}}
\newcommand{\fb}[1]{\textbf{#1}}

\begin{document}

\thispagestyle{empty}
\begin{center}

{\Large\bf Jacek Kobus\\[20pt]}

{\LARGE\bf 2 Dimensional Finite Difference \\[10pt]
Hartree--Fock Program}

\vskip 1cm

{\Large User's Guide}

\vskip 0.5cm

{\large version 2.2 }
\end{center}

\clearemptydoublepage
\tableofcontents
\thispagestyle{empty}
\clearemptydoublepage

%% ***************************************************************************
%% *                                                                         *
%% *   Copyright (C) 1996-2008 Jacek Kobus <jkob@fizyka.umk.pl>              *
%% *                 2018-     Susi Lehtola                                  *
%% *                                                                         *
%% *   This program is free software; you can redistribute it and/or modify  *
%% *   it under the terms of the GNU General Public License version 2 as     *
%% *   published by the Free Software Foundation.                            *
%% *                                                                         *
%% *   This program is distributed in the hope that it will be useful,       *
%% *   but WITHOUT ANY WARRANTY; without even the implied warranty of        *
%% *   MERCHANTABILITY or FITNESS FOR A PARTICULAR PURPOSE.  See the         *
%% *   GNU General Public License for more details.                          *
%% *                                                                         *
%% *   You should have received a copy of the GNU General Public License     *
%% *   along with this program; if not, write to the                         *
%% *   Free Software Foundation, Inc.,                                       *
%% *   59 Temple Place - Suite 330, Boston, MA  02111-1307, USA.             *
%% ***************************************************************************


\section{Description of input data}

x2dhf program accepts input data that consist of separate lines which contain
\begin{itemize}
\item[--] a label
\item[--] a label followed by a string of characters, integer(s)
  and/or real number(s)
\item[--] a string of characters, integer(s) and/or real number(s)
\end{itemize}
Real numbers can be written in a fixed-point or scientific notation.
\noindent
Note that
\begin{itemize}
\item[--] labels and strings can be in upper or lower case,

%% \item[--] compulsory labels, i.e. these that must be included in the
%%   input data file, are marked ``$\bullet$'' and the optional ones
%%   ``$\circ$'',

\item[--] the compulsory labels must follow the order given below; the
  optional ones can be~inserted anywhere between the \textbf{title}
  and \textbf{stop} labels,

\item[--] optional parameters are enclosed in square brackets,

\item[--] $r$ denotes a real number, $i$ -- an integer, $c$ -- a string
          of characters,

\item[--] an exclamation mark or a hash placed anywhere in an input
  line starts a comment and what follows ``!'' or ``\#'' is ignored.

\end{itemize}

\noindent
\subsection{Mandatory labels}

The following labels must be specified in the specified order:
\begin{description}
%\item[$\bullet$] \textbf{TITLE}
\item \textbf{TITLE}
\begin{description}
\item[Format:] \textbf{title} $c$ \\ $c$ is any string of up to 74
  characters describing the current case. This string is added as a
  header to a text file with a .dat extension that contains basic data
  identifying a given case, i.e. atomic numbers of nuclei, grid size
  and the number of electrons and orbital and exchange functions.
\end{description}

%\item[$\bullet$] \textbf{NUCLEI}
\item \textbf{NUCLEI}
\begin{description}
\item[Format:] \textbf{nuclei} $Z_A \;\;Z_B\;\; R \;\;[\; c\;]$ \\
  Set the nuclei charges and the bond length.

%% FIXME Konstatnty format of nuclear charges
\begin{itemize}
\item[$Z_A$:] nuclear charge of centre A (real)
\item[$Z_B$:] nuclear charge of centre B (real)
\item[$R$:] bond length (real)
\item[$c$:] $angstrom$ -- the internuclear separation can be given in
  \AA{}ngstr\"om if this string is included (the conversion factor
  0.529177249 is used)
\end{itemize}
  If $|Z_A-Z_B|<10^{-6}$ then the molecule is considered to be a
  homonuclear one (see \ft{setDefaults}).
%  and the inversion symmetry of the orbitals is enforced.
\end{description}

%\item[$\bullet$] \textbf{CONFIG}
\item \textbf{CONFIG}
\begin{description}
\item[Format:] \textbf{config} $i$
\begin{itemize}
\item[$i$:] the total charge of a system
\end{itemize}
\end{description}

The following cards define molecular orbitals and their occupation.
\textbf{Note that the last orbital description card must contain the
  $end$ label.}

The possible formats are:
\begin{description}
\item[Format:] $i\;\;\;c$
\begin{itemize}
\item[$i$:] number of fully occupied orbitals of a given irreducible
     representation (irrep) of the $C_{\infty\,v}$ group;
	two electrons make $\sigma$~orbitals fully occupied
     and four electrons -- orbitals of other symmetries
\item[$c$:] symbol of the $C_{\infty\,v}$ irrep to which the orbitals
     belong ({\it sigma, pi, delta} or~{\it phi})
\end{itemize}
\end{description}


\begin{description}
\item[Format:] $i\;\;\;c_1 \;\;c_2$
\begin{itemize}
\item[$i$:] number of fully occupied orbitals of a given irrep of the
$D_{\infty\,h}$ group
\item[$c_1$:] symbol of the $C_{\infty\,v}$ irrep to which the orbitals belong
       ({\it sigma, pi, delta} or~{\it phi})
\item[$c_2$:] symbol for the inversion symmetry of the $D_{\infty\,h}$ irreps
       ({\it u} or {\it g})
\end{itemize}
Use this format for a homonuclear molecule unless \textbf{break} card
is~included.
\end{description}


\begin{description}
\item[Format:]
$i\;\;\;c_1\;\;c_2\;\;[c_3\;\;[\;c_4\;\;[\;c_5\;]\;]\;]$
\begin{itemize}
\item[$i$:] number of orbitals of a given irrep of the $C_{\infty\,v}$  group
\item[$c_1$:] symbol for the $C_{\infty\,v}$ irreps to which the
orbitals belong ({\it sigma, pi, delta, phi})
\item[$c_2$-$c_5$:] $+,\;-$ or . (a dot); $+/-$ denotes spin up/down
electron and . denotes an unoccupied spin orbital
\end{itemize}
\end{description}

\begin{description}
\item[Format:] $i\;\;\;c_1\;\;c_2\;\;c_3\;\;[\;c_4\;\;[\;c_5\;\;[\;c_6\;]\;]\;]$
\begin{itemize}
\item[$i$:]  number of orbitals of a given irrep of the
$D_{\infty\,h}$ group
\item[$c_1$:] symbol for the $C_{\infty\,v}$ irrep to which the
orbitals belong ({\it sigma, pi, delta, phi})
\item[$c_2$:] symbol for the inversion symmetry of the $D_{\infty\,h}$
irrep ({\it u} or {\it g})
\item[$c_3$-$c_6$:] $+,\;-$ or . (a dot); $+/-$ denotes spin up/down
electron and . denotes an unoccupied spin orbital
\end{itemize}
\end{description}


%\item[$\bullet$] \textbf{GRID}
\item \textbf{GRID}

\begin{description}
\item[Format:] \textbf{grid} $N_{\nu}$ $R_{\infty}$ \\
An integer and a real define a single two-dimensional grid.
\begin{itemize}
\item[$N_{\nu}$:] the number of grid points in $\nu$ variable
\item[$R_{\infty}$:] the practical infinity
\end{itemize}
$N_{\mu}$ is calculated so as to make the step size in $\mu$ variable equal to the
step size in $\nu$ variable. $N_{\nu}$ and $N_{\mu}$ have to~meet special conditions and if
they are not fulfilled the nearest (but smaller) appropriate values are used.
\end{description}

\begin{description}
\item[Format:] \textbf{grid} $N_{\nu}$ $N_{\mu}$ $R_{\infty}$ \\
Two integers and one real define a single two-dimensional grid.
\begin{itemize}
\item[$N_{\nu}$:] the number of grid points in $\nu$ variable
\item[$N_{\mu}$:] the number of grid points in $\mu$ variable
\item[$R_{\infty}$:] the practical infinity
\end{itemize}
This format may be needed when interpolation between grids is attempted.
\end{description}



%\newpage
%\item[$\bullet$] \textbf{ORBPOT}
\item \textbf{ORBPOT}
\begin{description}
\item[Format:] \textbf{orbpot} $c$\\
  where $c$ a character string determining the initial source of
  orbitals and potentials. Its allowed values are:
\begin{itemize}
\item $hydrogen$ -- molecular orbitals are formed as a linear combination of hydrogenic
  functions on centres $A$ and $B$ as defined via the \textbf{lcao} label. In the case of
  HF or HFS calculations Coulomb (exchange) potentials are approximated as a~linear
  combination of Thomas-Fermi ($1/r$) potentials at the two centres. If the OED method
  is~chosen the potential function is~approximated as a linear combination of $Z_A/r_1$
  and $Z_B/r_2$ terms and the exchange potentials are set to zero.

\item $gauss$ -- GAUSSIAN output is used to retrieve the basis set and
  the molecular orbital expansion coefficients. The GAUSSIAN output is
  assumed to be contained in gaussian.out and gaussian.pun
  files.\footnote{The program has been tested on output files produced
    by the 94, 03, 09 and 16 versions of the GAUSSIAN system of
    programs.} Coulomb and exchange potentials are initialized as in
  the \textsl{hydrogen} case; see \ft{prepGauss} for more details.
  To generate the necessary output from GAUSSIAN, you need to specify
  {\bf punch=mo gfinput} in the route section of the GAUSSIAN input
  file. When you've run GAUSSIAN, you should end up with a file {\bf
    fort.7} that contains the molecular orbitals. To run {\bf x2dhf},
  put the log file in the working directory as {\bf gaussian.out} and
  the molecular orbitals from {\bf fort.7} as {\bf gaussian.pun},
  which will be parsed by the program.

\item $old$ -- initial orbitals, Coulomb and exchange potentials are retrieved from disk
  files (2dhf\_\-input.\-orb, 2dhf\_\-input.\-coul and 2dhf\_input.exch, respectively) created in
  a previous run. Data defining the case are retrieved from a 2dhf\_input.\-dat textfile.

\item $qrhf$ -- radial Hartree--Fock orbitals for the centre A and B
  obtained from the qrhf program\footnote{J. Kobus and Ch. Froese
    Fischer, \textsl{Quasi-Relativistic Hartree--Fock program for
      Atoms}, to be published.}  are retrieved from disk files
  1dhf\-\_centreA.\-orb and 1dhf\-\_centreB.\-orb, respectively, and
  Coulomb and exchange potentials are initialized as in the
  \textsl{hydrogen} case (see routine \ft{initHF} for details).

\item $noexch$ -- orbitals and Coulomb potentials are retrieved from disk files and
  exchange potentials are initialized as in $hydrogen$ case; this is useful when going
  from DFT/HFS to HF calculations.

\item $nodat$ -- initial orbitals and potentials are retrieved from disk files but the
  content of a 2dhf\_input.dat file is retrieved from a 2dhf\_input.orb (binary) file. Use
  this value when reading binary data generated by the earlier than 2.0 versions of the
  program.

\end{itemize}
\end{description}



%\item[$\bullet$] \textbf{STOP}
\item \textbf{STOP}
\begin{description}
\item[Format:] \textbf{stop}\\
This label indicates the end of input data.
\end{description}

\end{description}


\subsection{Optional labels}
% FIXME
%\item[$\bullet$] \textbf{METHOD}

The following additional labels can be specified in any order:
\begin{description}


\item \textbf{BREAK}
\begin{description}
\item[Format:] \textbf{break} \\ When this label is present homonuclear molecules are
  calculated in $C_{\infty\,v}$ symmetry and the $D_{\infty\,h}$ symmetry labels ({\it u}
  or {\it g}) are redundant.

\end{description}


\item \textbf{CONV}
\begin{description}
\item[Format:] \textbf{conv} $[\;i_1\;[\;i_2\;[\;i_3\;]\;]\;]$\\ Sometimes the requested
  accuracy of a solution is set too high and cannot be satisfied on a~selected grid. As
  a~result SCF/SOR iteration process may continue in vain. To save CPU time the iterations
  are stopped if orbital energies or orbital norms display no improvement over the $i_2$
  and $i_3$ most recent iterations, respectively (20 by default). This mechanism is
  activated after $i_1$~initial iterations (600 by default).
\end{description}

\item \textbf{DEBUG}
\begin{description}
\item[Format:] \textbf{debug} $[\; i_1 \; [\;i_2 \ldots \;] \;i_{40}\;]$ \\
  Up to 40 different debug flags can be set at a time.
%% \footnote{The maximum number of flags can be changed by adjusting the parameter
%% \texttt{maxflags} (see \ft{inputData}).}.  %%%%%%%%%%%%%%%%%%%%%%%%%%%%%%%%%%%%%%%%%%
  If the integer $i_k$ is present the debug flag $i_k$ is set, i.e. idbg$(i_k)=1$ $(1 \leq
  i_k<999)$. These are used to generate additional debugging information by adding the
  lines of the form
  \begin{verbatim}
       if (idbg(ik).eq.1) then
           print *, ``debugging something ...''
           ...
      endif
    \end{verbatim}
%These are used to generate debug information.
%If a debug flag is set to 500 an additional card is read in with
%$incrni$ and $incrmu$ variables which are
%parameters of PMTX routine used to print matrices (every $incrni$ row
%and every $incrmu$ column is printed.
\end{description}


\item \textbf{DFT}
\begin{description}
\item[Format:] \textbf{dft}  $[\;c_1\;]\;\; [\;c_2\;]$ \\
\begin{itemize}
\item[$c_1:$] specifies the type of DFT exchange potential to be used in
  Fock equations
\begin{itemize}
\item $c_1=lda$ -- the local density approximation with the potential
\begin{equation*}
         V_X(\alpha)=-\frac{3}{2} \alpha \left(\frac{3}{\pi}\right)^{1/3} 2^{-2/3}
                \sum_{\sigma} \rho_{\sigma}^{1/3}
\end{equation*}
where $\alpha$ is by default set to 2/3 (the Slater exchange
potential). To change this value use the \textbf{xalpha} label.

\item $c_1=b88$ -- the Becke exchange potential

\end{itemize}

\item[$c_2:$] selects the type of correlation potential to be used in
  Fock equations

\begin{itemize}
\item $c_2=lyp$ -- the correlation potential of Lee, Yang and Parr

\item $c_2=vwn$ -- the correlation potential of Vosko, Wilk and Nusair
\end{itemize}
\end{itemize}
\end{description}
When the bare label is present and the method selected is HF then the
exchange contributions (LDA, B88, PW86 and PW91) and the correlation
contributions (LYP and VWN) to the total energy are calculated upon
completion of the SCF iterations.

\item \textbf{EXCHIO}
\begin{description}
\item[Format:] \textbf{exchio} $c_1$ $c_2$\\ These parameters specify how exchange
  potentials are to be read/written and manipulated (stored in random access memory). The
  program always keeps all orbitals and Coulomb potentials in the memory. If computer
  resources are adequate all exchange potentials can also be kept there. However, during
  the relaxation of a particular orbital only a fraction of~exchange potentials
  is~needed. Thus all exchange potentials can be kept on disk as separate files (named
  fort.31, fort.32, $\ldots$ during a run) and only relevant ones are being retrieved when
  necessary.

The possible values of $c_1$ and $c_2$ are \textsl{in-one},
\textsl{in-many}, \textsl{out-one} and \textsl{out-many}. Possible
combinations are
\begin{itemize}
\item \verb+exchio in-many out-many+ -- read exchange potentials as separate files and
  write them back as separate files

\item \verb+exchio in-one out-many+ -- read all exchange potentials in one file but write
  them out as separate files

\item \verb+exchio in-many out-one+ -- read all exchange potentials separately but write
  them out as a single file

\item \verb+exchio in-one out-one+ -- read and write exchange potentials in the form of a
  single file (same as $i_2=3$); this is the default
\end{itemize}

\end{description}


\item \textbf{FEFIELD}
\begin{description}
\item[Format:] \textbf{fefield} $r$
\begin{itemize}
\item[$r$:] a strength of an external static electric field directed along
  the internuclear axis (in atomic units)
\end{itemize}
\end{description}



\item \textbf{FERMI}
\begin{description}
\item[Format:] \textbf{fermi} $r_A$ $r_B$ \\ When this label is
  present the Fermi nuclear charge distribution is used. Optional
  parameters $r_A$ and $r_B$ define the atomic masses (in atomic mass
  units, amu) of nuclei A and B. If omitted the corresponding values are
  taken from the table of atomic masses compiled by Wapstra and Audi
  (see \ft{blk\_data}).
\end{description}


\item \textbf{FIXORB}
\begin{description}
\item[Format:] \textbf{fixorb} $[\; i_1 \; [\;i_2 \ldots \;] \;i_{40}\;]$ \\ This label is
  used to specify orbitals to be kept frozen during SCF/SOR process (they are not being
  renormalized nor orthogonalized during the process). $i_1$, $i_2$, $\ldots$ are the
  numbers of these orbitals as they appear on the program's listing, i.e. their order is
  reversed to that used when defining the electronic configuration (see the \fb{config}
  card). Up to 40 different orbitals can be set at a~time. Use the bare label to keep all
  orbitals frozen.

\end{description}

\item \textbf{FIXCOUL}
\begin{description}
\item[Format:] \textbf{fixcoul}\\ If this label is present then all
  Coulomb potentials are kept frozen during the SCF/SOR process.
\end{description}

\item \textbf{FIXEXCH}
\begin{description}
\item[Format:] \textbf{fixexch}\\ If this label is present then all
  exchange potentials are kept frozen during the SCF/SOR process.
\end{description}


\item \textbf{GAUSS}
\begin{description}
\item[Format:] \textbf{gauss} $r_A$ $r_B$ \\ When this label is present the Gauss nuclear
  charge distribution is used. Optional parameters $r_A$ and $r_B$ define the atomic
  masses (in amu) of nuclei A and B. If omitted, the corresponding values are taken from
  the table of atomic masses compiled by Wapstra and Audi (see \ft{blk\_data}).
\end{description}

\item \textbf{HOMO}
\begin{description}
\item[Format:] \textbf{homo}  \\
This label is used to impose explicitly $D_{\infty\,h}$ symmetry upon orbitals
of homonuclear molecules in order to improve SCF/SOR convergence.
\end{description}


\item \textbf{INOUT}
\begin{description}
\item[Format:] \textbf{inout} $c_1$ $c_2$\\ The x2dhf program can be compiled to support
  calculation using three different combinations of integer/real data types: i32 (4-byte
  integers, 8-byte reals), i64 (8-byte integers, 8-byte reals) and r128 (8-byte integers,
  16-byte reals); see src/Makefile.am for details.  Strings $c_1$ and $c_2$ determine the
  combination appropriate for the format of~input and output data, respectively, and each
  string can be i32, i64 or r128.

  In order to facilitate exchange of binary data generated on machines of~different
  architectures or using different compilers additional formats are available, namely
  i32f, i64f or r128f which allow to export/import data in formatted instead of
  unformatted form.
\end{description}


\item \textbf{INTERP}
\begin{description}
\item[Format:] \textbf{interp} \\ Use this label to change the grid
  between separate runs of the program.  Note that you can only change
  the grid in one direction at a time: either in the $\nu$ direction
  ($N_\nu$), or in the $\mu$ direction ($N_\mu$ and/or $R_\infty$).
\end{description}




\item \textbf{LCAO}
\begin{description}
\item[Format:] \textbf{lcao} $[\;i\;]$\\ If the source of orbitals is declared as
  \textsl{hydrogen} then this card must be present. In such a case the initialization of
  each of the orbitals has to be defined in~terms of the linear combination of atom
  centred hydrogen-like functions. For each orbital include a card of the following format
  (make sure that the order of orbitals should match the order specified under the
  \textbf{config} label):

\begin{description}
\item[Format:] $c_A\;\;n_A \;\;l_A \;\;\zeta_A \;\;\;\;c_B\;\;n_B\;\;l_B\;\;
  \zeta_B\;\;\;\;i_1\;\;\;\;[\;i_2\;]$ \\
where
\begin{itemize}
\item[] $c_A$ -- relative mixing coefficient for a hydrogenic orbital on the
                    $Z_A$ centre (real),
\item[] $n_A$ -- its principle quantum number (integer)
\item[] $l_A$ -- its orbital quantum number (integer)
\item[] $\zeta_A$ -- the effective nuclear charge if $i=1$ (default) or
a screening parameter if $i=2$ (real)
\item[] $c_B$  -- relative mixing coefficient for a hydrogenic orbital on the
                    $Z_B$ centre (real),
\item[] $n_B$ -- its principle quantum number (integer)
\item[] $l_B$ -- its orbital quantum number (integer)
\item[] $\zeta_B$ -- the effective nuclear charge if $i=1$ (default) or
a screening parameter if $i=2$ (real)
\item[] $i_1$ -- set to 1 to indicate that this orbital should be initialized as a linear
  combination of hydrogenic functions and not taken from a disk orbital file (integer). When
  the source of orbitals is declared as \textsl{hydrogen} these flags are ignored.
  When the source of orbitals is declared as \textsl{old} and
  the orbital and Coulomb potential data files contain fewer functions than defined by
  \textsl{config} card this flag can be used to indicate which orbitals are missing in
  orbital data files and require initialization. This can be useful when generating
  virtual orbitals for a potential formed from already given set of converged orbitals.

  In case when the source of orbitals is declared as \textsl{old} and all orbitals are
  already defined this flag can be used in DFT calculations to generate virtual orbitals
  for some sort of a local potential build from a given subset of orbitals that are kept
  frozen during the SCF process (see \fb{fixorb} label)

\item[] $i_2$ -- a number of successive over-relaxations for a given orbital
(integer); if omitted is set to 10
\end{itemize}
The mixing coefficients are normalized so that $|c_A|+|c_B|=1$

\end{description}
\end{description}


\item \textbf{MCSOR}
\begin{description}
\item[Format:] \textbf{mcsor} $[\;i_1\;[\;i_2\;]\;]$\\
Selects the MCSOR method for solving the Poisson equations for orbitals and potentials
(default) and changes the value of the MCSOR relaxation sweeps during a single SCF
cycle for orbitals ($i_1$) and potentials ($i_2$); by default $i_1=i_2=10$.

\end{description}
\item \textbf{METHOD}
\begin{description}
\item[Format:] \textbf{method} $c$\\
Select the type of calculation.
\begin{itemize}
\item[$c$:] HF -- the Hartree--Fock method
\item[$c$:] DFT -- the Hartree--Fock method with the $X\alpha$ exchange
  potential ($\alpha=2/3$); see the \fb{dft} label to choose
  another exchange or correlation potential

\item[$c$:] HFS -- the Hartree--Fock-Slater method (Hartree--Fock with the
  $X\alpha$ exchange potential) with an optimum value of the $\alpha$
  parameter (see \ft{blk-\-data.\-inc} for details)

\item[$c$:] OED -- One Electron Diatomic ground and excited states
  can be calculated for the Coulomb potential in the prolate
  spheroidal coordinates (default). It is also possible to specify
the Coulomb and Kramers-Henneberger potentials in cylindrical
coordinates (see the \fb{poth3}, \fb{potkh}, \fb{potharm2}, \fb{potharm3}  labels,
respectively). When more than one orbital is specified calculations
are carried out as if in the case of a multielectron
system.\footnote{In this type of calculations convergence rates differ
  greatly between orbitals. Therefore, if for a given orbital the
  orbital energy threshold is reached it is being frozen.}

\item[$c$:] SCMC -- the Hartree--Fock method with $X\alpha$ exchange
  where the $\alpha$ parameter is calculated according to the
  self-consistent multiplicative constant
  method\footnote{V. V. Karasiev and E. V. Lude\~{n}a,
    \textsl{Self-consistent multiplicative constant method for the
      exchange energy in density functional theory},
    Phys. Rev. A~\textbf{65} (2002)
    062510. \doi{10.1103/PhysRevA.65.062510}}

\end{itemize}
\end{description}


\item \textbf{MULTIPOL}
\begin{description}
\item[Format:] \textbf{multipol} $r$ [ $i$ ]
\begin{itemize}
\item[$r$:] if $r>0$ multipole moment expansion coefficients are recalculated when the
  maximum error in orbital energy is changed by $r$ (the default value is 1.15; see
  \ft{setDefaults}). When the maximum error is less than $r$ the coefficients are
  recalculated at least every $i_2$ SCF iterations as defined by the SCF label. To
  suppress recalculation of the coefficients set $r$ to a negative real number. This is
  useful when generating potentials from a set of fixed orbitals, e.g. from GAUSSIAN
  orbitals.

\item[$i$:] number of terms in the multipole expansion used to calculate
  boundary value for potentials ($2 \le i\le 8$ and the default is 4).
\end{itemize}
\end{description}


\item \textbf{OMEGA}
\begin{description}
\item[Format:] \textbf{omega} $\omega_{orb}$ [ $\omega_{pot}$ ] \\ One or two real numbers
  setting over-relaxation parameters for relaxation of orbitals and potentials. The
  negative value of $\omega_{orb}$/$\omega_{pot}$ indicates that its value should be
  set to a~near optimum value obtained from a~semiempirical formula (see \ft{initCBlocks}
  and \ft{setomega} for details).
\end{description}

\item \textbf{OMEGAOPT}
\begin{description}
\item[Format:] \textbf{omegaopt} [ i [ $r_{orb}$ [ $r_{pot}$ ] ] ] \\ Optional integer
  parameter can be set to 1 (default) or 2. In the former case rather conservative (safe)
  $\omega$ values are chosen (this is equivalent to using the omega card with the negative
  values of the parameters). In the latter case somewhat better values are chosen but
  faster convergence is to be expected only when good initial estimates of orbitals and
  potentials are available or when calculations with fixed orbitals or~potentials are
  performed. The near-optimal $\omega$ values obtained from a semi-empirical formula are
  scaled down to~produce final values used by the program. The default values of the
  scaling factors for the orbital and potential over-relaxation parameters (0.986 and
  0.997, respectively) can be changed by setting $r_{orb}$ and $r_{pot}$ to their desired
  values.
\end{description}

\item \textbf{ORDER}
\begin{description}
\item[Format:] \textbf{order} $[\;i\;]$\\ An integer defining the ordering of mesh points:
  1 -- natural column-wise, 2 -- 'middle' type of sweep (default), 3 -- natural
  row-wise, 4 -- reversed natural column-wise (see \ft{mesh} routine for details)
\end{description}



\item \textbf{POTGSZ}
\begin{description}
\item[Format:] \textbf{potgsz} \\ When the OED method is chosen then
  this label selects a model potential due to Green, Sellin and
  Zachor.\footnote{A. E. S. Green, D. L. Sellin and A. S. Zachor,
    \textsl{Analytic Independent-Particle Model for Atoms},
    Phys. Rev. \textbf{184} (1969) 1-9. \doi{10.1103/PhysRev.184.1}}
  For a given atom this potential produces HF-like orbitals but it has
  been found useful in finding decent starting orbitals for any
  molecular system.
\end{description}


\item \textbf{POTGSZG}
\begin{description}
\item[Format:] \textbf{potgszg} \\ When the OED method is chosen then
  this label selects a~model potential due to Green, Sellin and Zachor
  and the Gauss nuclear charge distribution. For a~given atom this
  potential produces HF-like orbitals but it has been found useful in
  finding decent starting orbitals for any molecular system.
\end{description}


\item \textbf{POTHARM2}
\begin{description}
\item[Format:] \textbf{potharm2} $\omega$ \\ When the OED method is chosen then this
  label selects a~two-dimensional model potential of the form $-\omega^2(x^2+y^2)$, where
  $\omega$ is a strength of the potential (real).%
  \footnote{This parameter $\omega$ should not be confused with the
    over-relaxation parameter of the SOR method.}.
\end{description}


\item \textbf{POTHARM3}
\begin{description}
\item[Format:] \textbf{potharm3} $\omega$ \\ When the OED method is chosen then this label
  selects a~three-dimensional model potential of the form $-\omega^2(x^2+y^2+z^2)$, where
  $\omega$ is a strength of the potential (real).
\end{description}



\item \textbf{POTH3}
\begin{description}
\item[Format:] \textbf{poth3} $m\;\;a\;\;V_0$ \\ When the OED method
  is chosen then this label selects a~two-dimensional model potential
  of the form $ -V_0/\sqrt{a^2+r^2}$ (see \ft{kh.f90} for
  details). The following parameters can be set

\begin{itemize}
\item [$m:$] magnetic quantum number of a state (integer)
\item [$a:$] width of the model potential (real)
\item [$V_0:$] depth of the model potential (real)
\end{itemize}
In order to get the hydrogen Coulomb potential set $a=0.0$ and
$V_0=1.0$. Set $a>0$ and $V>0$ to choose its smoothed variant.

\end{description}


\item \textbf{POTKH}
\begin{description}
\item[Format:] \textbf{potkh}
  $m\;\;\varepsilon\;\;\omega\;\;\;[\;a\;[\;V_0\;[\;N\;]\;]\;]$
  \\ When the OED method is chosen then this label selects the
  Kramers-Henneberger potential (see routine \ft{kh.f90} for
  details). The following parameters can be set
\begin{itemize}
\item [$m:$] magnetic quantum number of a state being calculated (integer)
\item [$\varepsilon:$] laser intensity (real)
\item [$\omega:$] laser cycle frequency (real)
\item [$a:$] original (before averaging over one laser cycle)
  soft-core potential width (a positive real number, by default $a=1.0$)
\item [$V_0:$] original soft-core potential depth (by default $V_0=1.0$)
\item [$N:$] number of intervals in the Simpson quadrature (an
  integer, $N=1000$ by default)
\end{itemize}
\end{description}


\item \textbf{PRINT}
\begin{description}
\item[Format:] \textbf{print} $[\; i_1 \; [\;i_2 \ldots \;] \;i_{40}\;]$ \\ Up to 40
  different printing flags can be set at a~time. If the integer $i_k$ is encountered the
  printing flag $i_k$ is set, i.e. iprint$(i_k)=1$ $(1 \leq i_k<999)$. These are used
  to~generate additional printouts by adding the lines of~the form
  \begin{verbatim}
       if (iprint(ik).eq.1) then
           print *, ``printing something ...''
           ...
      endif
    \end{verbatim}

Set
\begin{itemize}
\item $i_1=110$ to print a total radial density relative to the
  centre~A along the internuclear axis ($-R_{\infty}\le z\le -R/2$)

\item $i_2=111$ to print a total radial density relative to the
  centre~B along the internuclear axis ($R/2\le z\le R_{\infty}$)
\end{itemize}
See \ft{inputData} for a~list of available flags.

\end{description}


\item \textbf{PRTEVERY}
\begin{description}
\item[Format:] \textbf{prtevery} $i_1$ $i_2$ \\ Routine \ft{pmtx} can be used to output
  two-dimensional arrays in a~tabular row-wise form with every $i_1$-th row and $i_2$-th
  column being printed (by default every 10th row and column is selected)
\end{description}


\item \textbf{SCF}
\begin{description}
\item[Format:] \textbf{scf}
$[\;i_1\;[\;i_2\;[\;i_3\;[\;i_4\;[i_5]\;]\;]\;]\;]$
\begin{itemize}
\item[$i_1$:] maximum number of scf iterations (default 1000); to skip the scf step set
  $i_1$ to a negative integer,
\item[$i_2$:] every $i_2$ SCF iterations orbitals and potentials are saved to~disk
  (default 20).
  If $i_2=0$ functions are saved on disk upon completion of the SCF
  process. If~$i_2<0$ functions are never written to disk,
\item[$i_3$:] if the maximum error in orbital energy is less than $10^{-i_3}$ than the SCF
  process is terminated (10 by default),
\item[$i_4$:] if the maximum error in orbital norm is less than $10^{-i_4}$ than SCF
  process is terminated (10 by default),
\item[$i_5$:] the level of output during SCF process
\begin{itemize}
\item $i_5=1$ -- the orbital energy, the difference between its current and previous
  value, the normalization error and the (absolute) value of~the largest overlap integral
  between the current orbital and all the lower lying ones of the same symmetry (the value
  is zero for the lowest orbitals of each symmetry) is~printed for every orbital in every
  SCF iteration
\item $i_5=2$ -- the orbital energy, the difference between its current and previous value
  and the normalization error is~printed for the worst converged orbital in energy (first
  line) and norm (second line) in every SCF iteration (default)
\item $i_5=3$ -- the orbital energy, the difference between its current and previous values
  and the normalization error is~printed for the worst converged orbital in energy (first
  line) and norm (second line) every $i_2$ iterations. Printing of ``... multipole moment
  expansion coefficients (re)calculated ...''  communique is suppressed
\end{itemize}
\end{itemize}
Total energy is printed every $i_2$ iterations.
\end{description}


\item \textbf{SOR}
\begin{description}
\item[Format:] \textbf{sor} $[\;i_1\;[\;i_2\;]\;]$\\
Selects the SOR method for solving the Poisson equations for orbitals and potentials
(default) and changes the value of the SOR relaxation sweeps during a single SCF
cycle for orbitals ($i_1$) and potentials ($i_2$); by default $i_1=i_2=10$.

\end{description}

\item \textbf{XALPHA}
\begin{description}
\item[Format:] \textbf{xalpha} $\alpha$ \\ This label allows to change the $\alpha$
  parameter of the LDA potential (see the HFS/DFT method); by default $\alpha$ is set
  to 2/3.
\end{description}

\end{description}


\subsection{Deprecated labels}

The following labels have been replaced by others but are supported for backward
compatibility with the previous version of the input data:

\begin{description}
\item \textbf{INITIAL} (deprecated, use \textbf{ORBPOT and LCAO} instead)
\begin{description}
\item[Format:] \textbf{initial} $i_1\;[\;i_2\;[\;i_3\;]\;]$
\begin{itemize}
\item[$i_1$:] determine the initial source of orbitals and potentials:
\begin{itemize}
\item $i_1=1$ --   molecular orbitals are formed as a linear
  combination of hydrogenic functions on centres $A$ and $B$ (see
  the description of $i_3$ for further details);
  in the case of HF or HFS calculations Coulomb
  (exchange) potentials are approximated as a linear combination of
  Thomas-Fermi ($1/r$) potentials at the two centres; if method OED is
  chosen the potential function is approximated as a linear
  combination of $Z_A/r_1$ and $Z_B/r_2$ terms and the exchange
  potentials are set to zero

\item $i_1=2$ -- GAUSSIAN output is used to retrieve the basis set and
  the molecular orbital expansion coefficients. The GAUSSIAN output is
  assumed to be contained in gaussian.out and gaussian.pun files.
  Coulomb and exchange potentials are initialized as in $i_1=1$ case;
  see \ft{prepGauss} for more details.

\item $i_1=3$ -- GAUSSIAN customized output is used to retrieve
  exponents and expansion coefficients of molecular orbitals (it is
  assumed that the output is contained in gaussianc.out file) and
  Coulomb and exchange potentials are initialized as in $i_1=1$ case;
  see \ft{prepGaussCust} for details

\item $i_1=5$ -- initial orbitals, Coulomb and exchange potentials are
  retrieved from disk files (2dhf\_input.orb, 2dhf\_input.coul and
  2dhf\_input.exch, respectively) created in a previous run. Data
  defining the case are retrieved from a 2dhf\_input.\-dat text file.

\item $i_1=6$ -- orbitals and Coulomb potentials are retrieved from
  disk files and exchange potentials are initialized as in $i_1=1$
  case (convenient when going from HFS to HF calculations)

\item $i_1=11$ -- radial Hartree--Fock orbitals for the centre A and~B
  are retrieved from disk files 1dhf\_inputA.orb and 1dhf\_inputB.orb,
  respectively, and Coulomb and exchange potentials are initialized as
  in the \textsl{hydrogen} case (see routine \ft{initHF} for details).

\item $i_1=55$ -- initial orbitals and potentials are retrieved from disk files but the
  content of a 2dhf\_input.dat file is retrieved from a 2dhf\_input.orb (binary) file. Use
  this value when reading binary data generated by older versions of the program.

\end{itemize}

\item[$i_2$:] specifies how exchange potentials are to be read/written and manipulated
  (stored in random access memory). The program always keeps all orbitals and Coulomb
  potentials in the memory. If computer resources are adequate all exchange potentials can
  also be kept there. However, during the relaxation of a particular orbital only a
  fraction of~exchange potentials is~needed. Thus all exchange potentials can be kept on
  disk as separate files (named fort.31, fort.32, $\ldots$ during a run) and only relevant
  ones are being retrieved when necessary.
\begin{itemize}
\item $i_2=0$ -- read exchange potentials as separate files and
                      write them back as~separate files
\item $i_2=1$ -- read all exchange potentials in a file but write
                       them out as~separate files
\item $i_2=2$ -- read all exchange potentials separately but write
                    them out as a~single file
\item $i_2=3$ -- read and write exchange potentials in the form of
                    a~single file (default)
\end{itemize}

\item[$i_3$:] if $i_1=1$ then this parameter must be set to 1 or 2 (if
  omitted it is set to 1). In such a case the initialization of
  each of the orbitals has to be defined in terms of the linear
  combination of atom-centred hydrogen-like functions
  For each orbital include a card of the
  following format (make sure that the order of orbitals should match
  the order specified under the \textbf{config} label):
\begin{description}
\item[Format:] $c_A\;\;n_A \;\;l_A \;\;\zeta_A \;\;\;\;c_B\;\;n_B\;\;l_B\;\;
  \zeta_B\;\;\;\;i_1\;\;\;\;[\;i_2\;]$ \\
where
\begin{itemize}
\item[] $c_A$ -- relative mixing coefficient for a hydrogenic orbital on the
                    $Z_A$ centre (real),
\item[] $n_A$ -- its principle quantum number (integer)
\item[] $l_A$ -- its orbital quantum number (integer)
\item[] $\zeta_A$ -- the effective nuclear charge if $i_3=1$ or
a screening parameter if $i_3=2$ (real)
\item[] $c_B$  -- relative mixing coefficient for a hydrogenic orbital on the
                    $Z_B$ centre (real),
\item[] $n_B$ -- its principle quantum number (integer)
\item[] $l_B$ -- its orbital quantum number (integer)
\item[] $\zeta_B$ -- the effective nuclear charge if $i_3=1$ or
a screening parameter if $i_3=2$ (real)
\item[] $i_1$ -- set to 1 to freeze the orbital during SCF; otherwise
  set to 0 (integer)
\item[] $i_2$ -- a number of successive over-relaxations for a given orbital
(integer); if omitted is set to 10
\end{itemize}

For other values of $i_1$ than 1 the orbital cards can be omitted but
then the $i_3$ parameter must be set to 0.

\end{description}
\end{itemize}
\end{description}

\item \textbf{FIX}
\begin{description}
\item[Format:] \textbf{fix}  $[\;i_1\;[\;i_2\;[\;i_3\;]\;]\;]$\\
  If $i_1$, $i_2$ or $i_3$ are set to 1 then orbitals, Coulomb
  potentials or~exchange potentials, respectively, are kept frozen during the
  SCF/SOR process (the respective default values are 0, 0 and 2).
  If $i_3=2$ then exchange potentials are relaxed only once during an
  SCF cycle.  $i_2$~and $i_3$~cannot be set to 1 if hydrogenic
    orbitals are used to initiate the orbitals.
\end{description}


\item \textbf{OMEGA}
\begin{description}
\item[Format:] \textbf{omega} \\
\hspace*{0.9cm}$\omega_{orb}$ \\
\hspace*{0.9cm}$\omega_{pot}$ \\ Two real numbers setting
over-relaxation parameters for relaxation of orbitals and potentials.
The negative value of a given parameter indicates that its value
should be set to a near optimum value obtained from a semi-empirical
formula (see \ft{initCBlocks} and \ft{setomega} for details).
\end{description}

\end{description}


\newpage

\section{Examples of input data}

\begin{enumerate}

\item $^2$S ground state of the Th$^{+89}$ one-electron system (see
  {examples/\-oed/\-th+89/\-th+89\-\_1s.\-lst}).

\verbatiminput{../examples/oed/th+89/th+89_1s.data}



\item First excited $^2$S state of the Th$^{+89}$ one-electron
  system (see {examples/\-oed/\-th+89/\-th+89\-\_2s.\-lst}).

\verbatiminput{../examples/oed/th+89/th+89_2s.data}

\newpage

\item \label{example-be} Hartree--Fock ground state of the beryllium atom calculated in
  two consecutive steps (see examples/\-be/\-be.lst and examples/\-be/\-be-1.lst).

\verbatiminput{../examples/be/be.data}


\verbatiminput{../examples/be/be-1.data}


\newpage


\item Hartree--Fock ground state energy of the hydrogen molecule (see
  examples/h2/h2a.lst).

\verbatiminput{../examples/h2/h2a.data}

\item \label{example-bf} Hartree--Fock ground state of the BF molecule (see
  examples/bf/bf\_g94.lst), initialized with orbitals from GAUSSIAN.

\verbatiminput{../examples/bf/bf_g94.data}

\newpage
\item HF calculations for the lowest $^3P$ state of the carbon atom
  (see examples/c/c.lst).

\verbatiminput{../examples/c/c.data}

\item HF calculations for the lowest $^2P$ state of the $C^+$ ion (see
  examples/c+/c+.lst).

\verbatiminput{../examples/c+/c+.data}

\newpage
\item \label{example-c2a} HF calculations for the lowest state of the $C_2$ molecule (see
  examples/c2/c2a.lst and examples/c2/c2b.lst).

\verbatiminput{../examples/c2/c2a.data}

\verbatiminput{../examples/c2/c2b.data}

\item HF calculations for the lowest state of the $N_2$ molecule (see
  examples/n2/n2.lst).


\verbatiminput{../examples/n2/n2.data}

\newpage
\item HF calculations for the lowest state of the $F_2$ molecule (see
  examples/f2/f2.lst).

\verbatiminput{../examples/f2/f2.data}

\newpage
\item A series of HF calculations for the FH molecule in an external
  static electric field.

\begin{enumerate}
\item no external field (see examples/fh/fh-0.lst)

\verbatiminput{../examples/fh/fh-0.data}


\newpage

\item field strength -0.0001 a.u. (see examples/fh/fh-m1.lst)

\verbatiminput{../examples/fh/fh-m1.data}

\item field strength -0.0002 a.u. (see examples/fh/fh-m2.lst)

\verbatiminput{../examples/fh/fh-m2.data}

\newpage

\item field strength +0.0001 a.u. (see examples/fh/fh-p1.lst)

\verbatiminput{../examples/fh/fh-p1.data}

\item field strength 0.0002 a.u. (see examples/fh/fh-p2.lst)

\verbatiminput{../examples/fh/fh-p2.data}
\end{enumerate}

\newpage

\item DFT calculations with B88 and B88-LYP functionals (see
  examples/\-dft/\-be-1.\-lst and examples/\-dft/\-be-2.\-lst).

\verbatiminput{../examples/dft/be-1.data}

\verbatiminput{../examples/dft/be-2.data}


\newpage

\item Two lowest states of the 2D hydrogenic harmonic potential (see
  examples/\-oed/\-h3/\-h3-1.lst and examples/\-oed/\-h3/\-h3-2.lst).

\verbatiminput{../examples/oed/h3/h3-1.data}

\verbatiminput{../examples/oed/h3/h3-2.data}


\newpage

\item Lowest state of the Kramers-Henneberger potential (see
  examples/\-oed/\-kh/\-kh.lst).

\verbatiminput{../examples/oed/kh/kh.data}

\item SCMC ground state calculations for the beryllium atom (see
  examples/\-scmc/\-be-scmc.lst).

\verbatiminput{../examples/scmc/be-scmc.data}

\end{enumerate}

\newpage

\section{Program's data files}

There are several standard names used by the program to keep track of its input and output
disk files. Normally the program writes out the data in the course of computations and
upon its completion into the following disk files:
\begin{description}

\item \texttt{2dhf\_output.dat} (a text (ASCII) file) containing the title of a case, the
  time and date of its commencement, the number of mesh points, the internuclear distance,
  the charges of nuclei and the number of orbitals, electrons and exchange potentials (see
  \texttt{writeDisk} for details),

\item \texttt{2dhf\_output.orb} (a binary file) containing molecular orbitals (in the
  order specified by the input data following \textbf{config} label) followed by their
  normalization factors, orbital energies, Lagrange multipliers and multipole moment
  expansion coefficients (see \texttt{write\*}),

\item \texttt{2dhf\_output.coul} (a binary file) containing
  corresponding Coulomb potentials and

\item \texttt{2dhf\_output.exch} (a binary file) containing all exchange potentials if the
  \textbf{exchio [in-one$|$in-many] out-one} card is present

\item \texttt{fort.31, fort.32, \ldots} (binary files) each containing the exchange
  potential required for a particular pair of orbitals if the \textbf{exchio in-many
    [out-one$|$out-many]} or \textbf{exchio [in-one$|$in-many] out-many} card is present
\end{description}

These files can be used to restart a given case or run another with slightly modified
parameters. If \textbf{orbpot old} card is present orbitals are retrieved from
\texttt{2dhf\-\_in\-put\-.orb} file, Coulomb potentials from \texttt{2dhf\_input.coul} and
exchange potentials from \texttt{2dhf\_input.exch} file (or \texttt{fort.31}, \texttt{fort
  32}, \ldots files, if \textbf{exchio in-many [out-one$|$out-many]}). Note that there
is~only one set of \texttt{fort} files.

\bigskip

\section{How to run the program?}

\noindent
In order to simplify the usage of the program, the xhf script (see tests/xhf) is~provided
to~facilitate handling of the disk files.  The command xhf can be envoked with one, two or
three parameters. There are two basic modes of its usage:



\begin{description}
\item \hspace*{0.5cm} \texttt{./xhf file1 file2}\\ runs x2dhf reading
  input data from \texttt{file1.data} file and writing text data
  describing the case into \texttt{file2.dat} file and binary data
  with orbitals and potentials into \texttt{file2.orb},
  \texttt{file2.coul} and \texttt{file2.exch} files.
\end{description}


\begin{description}
\item \hspace*{0.5cm} \texttt{./xhf file1 fil2 file3}\\ runs x2dhf
  reading input data from \texttt{file1.data} and initial orbitals and
  potentials from \texttt{file2.dat}, \texttt{file2.orb},
  \texttt{file2.coul} and \texttt{file2.exch} files and writing
  resulting data into \texttt{file3.dat}, \texttt{file3.orb},
  \texttt{file3.coul} and \texttt{file3.exch} files.
\end{description}

If, for example, \texttt{be.data} file contains input data for
the beryllium atom (see Example~\ref{example-be}) then
\begin{description}
\item \hspace*{0.5cm} \texttt{./xhf be be-1}
\end{description}
starts and performs the first 300 SCF iterations. Type
\begin{description}
\item \hspace*{0.5cm} \texttt{./xhf be-1 be-1 be-2}
\end{description}
to continue calculations. In order to converge the SCF process
even better increase the convergence parameters (see the \textbf{scf}
label) and use the following
command
\begin{description}
\item \hspace*{0.5cm} \texttt{./xhf be-1 be-2 be-1}
\end{description}



In addition, the xhf script can be used to perform the following
tasks:

\begin{description}
\item \hspace*{0.5cm} \texttt{./xhf stop}\\ creates a
  \texttt{stop\_x2dhf} file in a current directory to stop a running
  calculation % (see Section~\ref{section:hints})
\end{description}

\begin{description}
\item \hspace*{0.5cm} \texttt{./xhf mkgauss filename}\\ creates
  symbolic links \texttt{gaussian.out} and \texttt{gaussian.pun} to
  files \texttt{filename.\-out} and \texttt{filename.pun},
  respectively (see e.g. Example~\ref{example-bf}).
\end{description}

\begin{description}
\item \hspace*{0.5cm} \texttt{./xhf rmgauss}\\ removes
  \texttt{gaussian.out} and \texttt{gaussian.pun} files from a current
  directory
\end{description}

\begin{description}
\item \hspace*{0.5cm} \texttt{./xhf clean} \\ removes
  \texttt{*.[dat|orb|coul|exch]} files
\end{description}

Use \ft{./xhf -h|help} to get the complete list of available options.
%\newpage

\section{Useful hints}

\begin{enumerate}
\item The program should be easy to use provided you can start a
  calculation for a specific system. You should not encounter any
  serious problems when the system contains atoms from the first two
  rows of the Periodic Table. Then even the rough hydrogenic estimates
  of the orbitals should prove adequate and after the initial couple
  of dozen of iterations a smooth convergence should set in.

  If, however, a system contains more than 15-20 electrons the initial
  estimates of the orbitals have to be good enough to avoid
  divergences. Then, you have to choose the parameters of the
  hydrogenic orbitals carefully or perform the finite basis set
  calculations using the GAUSSIAN program to provide the
  initialization data for orbitals (see Example~\ref{example-bf}).

  One can also use HFS method to produce initial estimates of orbitals
  and Coulomb potentials. For example, to start calculations for the
  neon atom one can use the following input data:

  \verbatiminput{../examples/ne/ne-a.data}

  This input produces good enough HFS initial estimates so that the
  calculations can be continued at the HF level with the corresponding
  input:

  \verbatiminput{../examples/ne/ne-b.data}

  One can also use HF method with some model potential, e.g. the model
  potential of Green, Sellin, Zachor (label POTGSZ).

\item Choose smaller values of the relaxation parameters ($1.7\leq
  \omega\leq 1.85$) to avoid divergences in the first few dozens of
  SCF iterations (values as small as 0.2 may be used for very heavy
  systems). Set the maximum number of SCF iterations to a small value
  -- between 20 and 500 -- and/or impose crude convergence criteria
  for the orbital energy and normalization.

\item Once the orbitals and potentials have reached initial
  convergence, the relaxation parameters should be increased to their
  (near) optimum value (see the \textbf{omega} label and
  Example~\ref{example-c2a}).

  It is possible to set $\omega_{pot}$ to its near-optimal value by
  calculating it from a~semi-empirical formula; see the \ft{omega}
  label. As a rule of thumb the optimal value of the orbital
  relaxation parameter is somewhat smaller and, by~default,
  is~obtained by scaling the $\omega_{pot}$ value by 0.98 (see
  \ft{setDefaults}).


\item In case of convergence problems try to perform calculations on a
  sparser grid. For example, the $[61\times 79/30]$ grid is
  sufficient to check the quality of the initial data for the Ne$_2$
  system.

\item \label{section:hints} How to stop the program gracefully during
  a lengthy calculation without killing the process and interrupting
  disk read/write operations?  All you have to do is to create a (zero
  length) file named \texttt{stop\_x2dhf} in a working directory by
  typing \texttt{./xhf stop} (you can also use the Unix \texttt{touch}
  command to~this end). The program stops whenever this file is
  detected upon the completion of a~current orbital/potential
  relaxation.



\end{enumerate}
\end{document}
